% -*-latex-*-
% 
% For questions, comments, concerns or complaints:
% thesis@mit.edu
% 
%
% $Log: cover.tex,v $
% Revision 1.9  2019/08/06 14:18:15  cmalin
% Replaced sample content with non-specific text.
%
% Revision 1.8  2008/05/13 15:02:15  jdreed
% Degree month is June, not May.  Added note about prevdegrees.
% Arthur Smith's title updated
%
% Revision 1.7  2001/02/08 18:53:16  boojum
% changed some \newpages to \cleardoublepages
%
% Revision 1.6  1999/10/21 14:49:31  boojum
% changed comment referring to documentstyle
%
% Revision 1.5  1999/10/21 14:39:04  boojum
% *** empty log message ***
%
% Revision 1.4  1997/04/18  17:54:10  othomas
% added page numbers on abstract and cover, and made 1 abstract
% page the default rather than 2.  (anne hunter tells me this
% is the new institute standard.)
%
% Revision 1.4  1997/04/18  17:54:10  othomas
% added page numbers on abstract and cover, and made 1 abstract
% page the default rather than 2.  (anne hunter tells me this
% is the new institute standard.)
%
% Revision 1.3  93/05/17  17:06:29  starflt
% Added acknowledgements section (suggested by tompalka)
% 
% Revision 1.2  92/04/22  13:13:13  epeisach
% Fixes for 1991 course 6 requirements
% Phrase "and to grant others the right to do so" has been added to 
% permission clause
% Second copy of abstract is not counted as separate pages so numbering works
% out
% 
% Revision 1.1  92/04/22  13:08:20  epeisach

% NOTE:
% These templates make an effort to conform to the MIT Thesis specifications,
% however the specifications can change. We recommend that you verify the
% layout of your title page with your thesis advisor and/or the MIT 
% Libraries before printing your final copy.
\title{Managing Risk in Geothermal Exploration and Production for Commercial Energy Diversification}

\author{Robert Chadwick Holmes}
% If you wish to list your previous degrees on the cover page, use the 
% previous degrees command:
%       \prevdegrees{A.A., Harvard University (1985)}
% You can use the \\ command to list multiple previous degrees
%       \prevdegrees{B.S., University of California (1978) \\
%                    S.M., Massachusetts Institute of Technology (1981)}
\department{System Design and Management Program}

% If the thesis is for two degrees simultaneously, list them both
% separated by \and like this:
% \degree{Doctor of Philosophy \and Master of Science}
\degree{Master of Science in Engineering and Management}

% As of the 2007-08 academic year, valid degree months are September, 
% February, or June.  The default is June.
\degreemonth{September}
\degreeyear{2021}
\thesisdate{August 6, 2021}

%% By default, the thesis will be copyrighted to MIT.  If you need to copyright
%% the thesis to yourself, just specify the `vi' documentclass option.  If for
%% some reason you want to exactly specify the copyright notice text, you can
%% use the \copyrightnoticetext command.  
%\copyrightnoticetext{\copyright IBM, 1990.  Do not open till Xmas.}

% If there is more than one supervisor, use the \supervisor command
% once for each.
\supervisor{Aim\'e Fournier}{Research Scientist, Earth and Planetary Sciences}
\supervisor{Bryan R. Moser}{Academic Director, System Design and Management}

% This is the department committee chairman, not the thesis committee
% chairman.  You should replace this with your Department's Committee
% Chairman.
\chairman{Joan Rubin}{Executive Director, System Design and Management}

% Make the titlepage based on the above information.  If you need
% something special and can't use the standard form, you can specify
% the exact text of the titlepage yourself.  Put it in a titlepage
% environment and leave blank lines where you want vertical space.
% The spaces will be adjusted to fill the entire page.  The dotted
% lines for the signatures are made with the \signature command.
\maketitle

% The abstractpage environment sets up everything on the page except
% the text itself.  The title and other header material are put at the
% top of the page, and the supervisors are listed at the bottom.  A
% new page is begun both before and after.  Of course, an abstract may
% be more than one page itself.  If you need more control over the
% format of the page, you can use the abstract environment, which puts
% the word "Abstract" at the beginning and single spaces its text.

%% You can either \input (*not* \include) your abstract file, or you can put
%% the text of the abstract directly between the \begin{abstractpage} and
%% \end{abstractpage} commands.

% First copy: start a new page, and save the page number.
\cleardoublepage
% Uncomment the next line if you do NOT want a page number on your
% abstract and acknowledgments pages.
% \pagestyle{empty}
\setcounter{savepage}{\thepage}
\begin{abstractpage}
% $Log: abstract.tex,v $
% Revision 1.1  93/05/14  14:56:25  starflt
% Initial revision
% 
% Revision 1.1  90/05/04  10:41:01  lwvanels
% Initial revision
% 
%
%% The text of your abstract and nothing else (other than comments) goes here.
%% It will be single-spaced and the rest of the text that is supposed to go on
%% the abstract page will be generated by the abstractpage environment.  This
%% file should be \input (not \include 'd) from cover.tex.
%Geothermal provides a continuous, low greenhouse-gas emissions source of energy with enormous potential in the United States, both singularly or as part of a renewable energy portfolio. Although a small contributor to the current national energy grid, geothermal capture for generating electricity dates back nearly a century for natural hydrothermal systems. More recently, technologies at various readiness levels give the promise of geothermal access using enhanced geothermal systems (EGS), which provide engineered solutions for subsurface fluid circulation to tap into thermal reservoirs in a wider variety of locations. Nevertheless, the risk of high costs associated with exploration and production remain a hurdle to broader adoption of geothermal as part of a diverse commercial energy mix.

%In this thesis, risk-mitigation strategies for geothermal exploration and production target two separate aspects of the system lifecycle. The first considers how data collected for interrelated earth systems can indicate geothermal potential at the play and prospect scale. Analytical workflows integrating geologic and geophysical data are used to estimate the subsurface geothermal gradient, with quantitative uncertainty estimates associated with the data inputs, the modeling approach, and the size of the solution space. These uncertainty estimates provide a measure of risk, as well as decision tools for investments in additional data-gathering activities before the first well is drilled. The second focus looks at flexibility in engineering design with real options for expanding an existing power facility with geothermal. Specifically, key uncertainties are defined and integrated into a cost-modeling approach that uses decision rules to define an ensemble of possible outcomes. Tailoring the model and decision rules to the potential field and location of interest allows for a rapid but thorough test of project feasibility and the selection of build-out alternatives that limit downside risk and capture upside potential. In total, the learnings from these investigations provide insights into how geothermal can be a commercially viable, low-carbon option as energy companies navigate the ongoing energy transition.
Geothermal provides a continuous, low-emissions source of energy with enormous potential in the United States, both singularly or as part of a broader energy mix. Although a small contributor to the current national energy grid, geothermal electricity generation dates back nearly a century for natural hydrothermal systems. More recently, enhanced geothermal systems (EGS) promise a broader reach with engineered solutions for extracting subsurface heat from a wider variety of locations. The potential synergy between the oil \& gas and geothermal offers an opportunity for building a lower-carbon energy portfolio that requires compatible skills and expertise. Nevertheless, the risks involved at multiple stages of the field lifecycle remain a hurdle to adoption of geothermal.

In this thesis, risk-mitigation strategies for geothermal target two phases of the lifecycle: exploration and production. The first strategy uses a diverse set of measurements spanning multiple interrelated earth systems to collectively determine geothermal potential at the play scale. Analytical workflows integrate geologic, geochemical, and geophysical data to estimate subsurface geothermal gradient, with quantitative uncertainty estimates associated with the measurements, the models, and the solution space. These uncertainty estimates provide a measure of risk, as well as decision tools for investments in additional data-gathering activities before the first well is drilled. The second strategy applies flexibility in engineering design to a hypothetical EGS expansion of an existing power facility. Specifically, key uncertainties are integrated into a cost model with operational decision rules to create an ensemble of possible outcomes. Tailoring the model and decision rules to a particular facility concept allows for a rapid feasibility testing and optimization of project actions that limit downside risk while capturing upside potential. Both of these strategies use uncertainty characterization to reduce the threat of high-consequence geothermal risks. And by including them in a broader risk management approach, oil \& gas companies can make data-driven decisions on investing in geothermal during the energy transition.
\end{abstractpage}

% Additional copy: start a new page, and reset the page number.  This way,
% the second copy of the abstract is not counted as separate pages.
% Uncomment the next 6 lines if you need two copies of the abstract
% page.
% \setcounter{page}{\thesavepage}
% \begin{abstractpage}
% % $Log: abstract.tex,v $
% Revision 1.1  93/05/14  14:56:25  starflt
% Initial revision
% 
% Revision 1.1  90/05/04  10:41:01  lwvanels
% Initial revision
% 
%
%% The text of your abstract and nothing else (other than comments) goes here.
%% It will be single-spaced and the rest of the text that is supposed to go on
%% the abstract page will be generated by the abstractpage environment.  This
%% file should be \input (not \include 'd) from cover.tex.
%Geothermal provides a continuous, low greenhouse-gas emissions source of energy with enormous potential in the United States, both singularly or as part of a renewable energy portfolio. Although a small contributor to the current national energy grid, geothermal capture for generating electricity dates back nearly a century for natural hydrothermal systems. More recently, technologies at various readiness levels give the promise of geothermal access using enhanced geothermal systems (EGS), which provide engineered solutions for subsurface fluid circulation to tap into thermal reservoirs in a wider variety of locations. Nevertheless, the risk of high costs associated with exploration and production remain a hurdle to broader adoption of geothermal as part of a diverse commercial energy mix.

%In this thesis, risk-mitigation strategies for geothermal exploration and production target two separate aspects of the system lifecycle. The first considers how data collected for interrelated earth systems can indicate geothermal potential at the play and prospect scale. Analytical workflows integrating geologic and geophysical data are used to estimate the subsurface geothermal gradient, with quantitative uncertainty estimates associated with the data inputs, the modeling approach, and the size of the solution space. These uncertainty estimates provide a measure of risk, as well as decision tools for investments in additional data-gathering activities before the first well is drilled. The second focus looks at flexibility in engineering design with real options for expanding an existing power facility with geothermal. Specifically, key uncertainties are defined and integrated into a cost-modeling approach that uses decision rules to define an ensemble of possible outcomes. Tailoring the model and decision rules to the potential field and location of interest allows for a rapid but thorough test of project feasibility and the selection of build-out alternatives that limit downside risk and capture upside potential. In total, the learnings from these investigations provide insights into how geothermal can be a commercially viable, low-carbon option as energy companies navigate the ongoing energy transition.
Geothermal provides a continuous, low-emissions source of energy with enormous potential in the United States, both singularly or as part of a broader energy mix. Although a small contributor to the current national energy grid, geothermal electricity generation dates back nearly a century for natural hydrothermal systems. More recently, enhanced geothermal systems (EGS) promise a broader reach with engineered solutions for extracting subsurface heat from a wider variety of locations. The potential synergy between the oil \& gas and geothermal offers an opportunity for building a lower-carbon energy portfolio that requires compatible skills and expertise. Nevertheless, the risks involved at multiple stages of the field lifecycle remain a hurdle to adoption of geothermal.

In this thesis, risk-mitigation strategies for geothermal target two phases of the lifecycle: exploration and production. The first strategy uses a diverse set of measurements spanning multiple interrelated earth systems to collectively determine geothermal potential at the play scale. Analytical workflows integrate geologic, geochemical, and geophysical data to estimate subsurface geothermal gradient, with quantitative uncertainty estimates associated with the measurements, the models, and the solution space. These uncertainty estimates provide a measure of risk, as well as decision tools for investments in additional data-gathering activities before the first well is drilled. The second strategy applies flexibility in engineering design to a hypothetical EGS expansion of an existing power facility. Specifically, key uncertainties are integrated into a cost model with operational decision rules to create an ensemble of possible outcomes. Tailoring the model and decision rules to a particular facility concept allows for a rapid feasibility testing and optimization of project actions that limit downside risk while capturing upside potential. Both of these strategies use uncertainty characterization to reduce the threat of high-consequence geothermal risks. And by including them in a broader risk management approach, oil \& gas companies can make data-driven decisions on investing in geothermal during the energy transition.
% \end{abstractpage}

\cleardoublepage

\section*{Acknowledgments}

The past year will be long remembered for the impact a global pandemic had on society at large. This thesis is a product of that time. Those mentioned below played a significant role in keeping things on track in spite of the many months of mask wearing; hand sanitizing; regular virus testing; quarantining; remote classes, meetings, and connection-building; and eventual vaccination. Some I have even met in person, although not all. That is all part of the legacy of this most unusual and memorable year.

[SDM]

[ERL]

[Chevron]

[Chevron cohort - both years]

[Family]

%%%%%%%%%%%%%%%%%%%%%%%%%%%%%%%%%%%%%%%%%%%%%%%%%%%%%%%%%%%%%%%%%%%%%%
% -*-latex-*-
