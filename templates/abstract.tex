% $Log: abstract.tex,v $
% Revision 1.1  93/05/14  14:56:25  starflt
% Initial revision
% 
% Revision 1.1  90/05/04  10:41:01  lwvanels
% Initial revision
% 
%
%% The text of your abstract and nothing else (other than comments) goes here.
%% It will be single-spaced and the rest of the text that is supposed to go on
%% the abstract page will be generated by the abstractpage environment.  This
%% file should be \input (not \include 'd) from cover.tex.
Geothermal provides a continuous, low greenhouse gas emissions source of energy with enormous potential in the United States, both singularly or as part of a renewable energy portfolio. Although a small contributor to the current national energy grid, geothermal capture for generating electricity dates back nearly a century for natural hydrothermal systems. More recently, technologies at various readiness levels give the promise of geothermal access using enhanced geothermal systems (EGS), which provide engineered solutions for subsurface fluid circulation to tap into thermal reservoirs in a broader variety of locations. Nevertheless, the risk of high costs associated with exploration and production remain a hurdle to broader adoption of geothermal as part of a diverse commercial energy mix.

In this thesis, mitigation strategies for geothermal exploration and production target two separate aspects of the system lifecycle. The first considers how data collected for interrelated earth systems can indicate geothermal potential at the play and prospect scale. Analytical workflows integrating geologic and geophysical data are used to estimate the subsurface geothermal gradient, with quantitative uncertainty estimates associated with the data inputs, the modeling approach, and the size of the solution space. These uncertainty estimates provide a measure of risk, as well as decision tools for investments in additional data-gathering activities before the first well is drilled. The second focus looks at flexibility in engineering design with real options for expanding an existing power facility with geothermal. Specifically, key uncertainties are defined and integrated into a cost modeling approach that uses decision rules to define an ensemble of possible outcomes. Tailoring the model and decision rules to the potential field and location of interest allows for a rapid but thorough test of project feasibility and the selection of build-out alternatives that limit downside risk and capture upside potential. In total, the learnings from these investigations provide insights into how geothermal can be a commercially-viable option as companies providing more carbon-intensive sources of energy navigate the on-going energy transition.