% -*- Mode:TeX -*-

%% IMPORTANT: The official thesis specifications are available at:
%%            http://libraries.mit.edu/archives/thesis-specs/
%%
%%            Please verify your thesis' formatting and copyright
%%            assignment before submission. If you notice any
%%            discrepancies between these templates and the 
%%            MIT Libraries' specs, please let us know
%%            by e-mailing thesis@mit.edu

%% The documentclass options along with the pagestyle can be used to generate
%% a technical report, a draft copy, or a regular thesis. You may need to
%% re-specify the pagestyle after you \include cover.tex. For more
%% information, see the first few lines of mitthesis.cls. 

%\documentclass[12pt,vi,twoside]{mitthesis}
%%
%%  If you want your thesis copyright to you instead of MIT, use the
%%  ``vi'' option, as above.
%%
\documentclass[%draft,
12pt,vi,twoside,leftblank]{mitthesis}
%%
%% If you want blank pages before new chapters to be labelled ``This
%% Page Intentionally Left Blank'', use the ``leftblank'' option, as
%% above. 

%\documentclass[12pt,twoside]{mitthesis}
\usepackage{lgrind}

%% for figures
\usepackage[utf8]{inputenc}
\usepackage{graphicx}
\graphicspath{{templates/images/}}
\usepackage{wrapfig}
\usepackage{caption}

%% for tables
\usepackage[normalem]{ulem}
\useunder{\uline}{\ul}{}
\usepackage[table,xcdraw]{xcolor}
% If you use beamer only pass "xcolor=table" option, i.e. \documentclass[xcolor=table]{beamer}

%% These have been added at the request of the MIT Libraries, because
%% some PDF conversions mess up the ligatures.  -LB, 1/22/2014
\usepackage{cmap}
\usepackage[T1]{fontenc}
\pagestyle{plain}

%% for List of Acronyms
\usepackage[acronym,toc]{glossaries}
\makenoidxglossaries

%% stylize hyperlinks
\usepackage{hyperref}
\usepackage{xcolor}
\hypersetup{
    colorlinks = true,
    linkcolor = blue,
    citecolor = blue
}

%% APA citations
\usepackage{apacite}
\usepackage{natbib}

%% For equation symbols
\usepackage[
  separate-uncertainty = true,
  multi-part-units = repeat
]{siunitx}

\usepackage{enumitem}
\usepackage{xparse}
\ExplSyntaxOn

%% support multiple citations with page numbers
\makeatletter
\NewDocumentCommand{\multicitep}{m}
 {
  \NAT@open
  \mjb_multicitep:n { #1 }
  \NAT@close
 }
\makeatother

\seq_new:N \l_mjb_multicite_in_seq
\seq_new:N \l_mjb_multicite_out_seq
\seq_new:N \l_mjb_cite_seq

\cs_new_protected:Npn \mjb_multicitep:n #1
 {
  \seq_set_split:Nnn \l_mjb_multicite_in_seq { ; } { #1 }
  \seq_clear:N \l_mjb_multicite_out_seq
  \seq_map_inline:Nn \l_mjb_multicite_in_seq
   {
    \mjb_cite_process:n { ##1 }
   }
  \seq_use:Nn \l_mjb_multicite_out_seq { ;~ }
 }

\cs_new_protected:Npn \mjb_cite_process:n #1
 {
  \seq_set_split:Nnn \l_mjb_cite_seq { , } { #1 }
  \int_compare:nTF { \seq_count:N \l_mjb_cite_seq == 1 }
   {
    \seq_put_right:Nn \l_mjb_multicite_out_seq
     { \citeauthor{#1},~\citeyear{#1} }
   }
   {
    \seq_put_right:Nx \l_mjb_multicite_out_seq
     {
      \exp_not:N \citeauthor{\seq_item:Nn \l_mjb_cite_seq { 1 }},~
      \exp_not:N \citeyear{\seq_item:Nn \l_mjb_cite_seq { 1 }},~
      \seq_item:Nn \l_mjb_cite_seq { 2 }
     }
   }
 }
\ExplSyntaxOff

%% This bit allows you to either specify only the files which you wish to
%% process, or `all' to process all files which you \include.
%% Krishna Sethuraman (1990).

%\typein [\files]{Enter file names to process, (chap1,chap2 ...), or `all' to process all files:}
\def\all{all}
\ifx\files\all \typeout{Including all files.} \else %\typeout{Including only \files.} \includeonly{\files} \fi

%% define list of acronyms
\section*{Table of Acronyms}
\newacronym{gcd}{GCD}{Greatest Common Divisor}
\newacronym{lcm}{LCM}{Least Common Multiple}

\gls{gcd} is fun like \gls{lcm}

\printglossary[title=List of Acronyms, toctitle=List of Acronyms, type=\acronymtype]

%% zotero integration
%\addbibresource{zotero}

\begin{document}

% -*-latex-*-
% 
% For questions, comments, concerns or complaints:
% thesis@mit.edu
% 
%
% $Log: cover.tex,v $
% Revision 1.9  2019/08/06 14:18:15  cmalin
% Replaced sample content with non-specific text.
%
% Revision 1.8  2008/05/13 15:02:15  jdreed
% Degree month is June, not May.  Added note about prevdegrees.
% Arthur Smith's title updated
%
% Revision 1.7  2001/02/08 18:53:16  boojum
% changed some \newpages to \cleardoublepages
%
% Revision 1.6  1999/10/21 14:49:31  boojum
% changed comment referring to documentstyle
%
% Revision 1.5  1999/10/21 14:39:04  boojum
% *** empty log message ***
%
% Revision 1.4  1997/04/18  17:54:10  othomas
% added page numbers on abstract and cover, and made 1 abstract
% page the default rather than 2.  (anne hunter tells me this
% is the new institute standard.)
%
% Revision 1.4  1997/04/18  17:54:10  othomas
% added page numbers on abstract and cover, and made 1 abstract
% page the default rather than 2.  (anne hunter tells me this
% is the new institute standard.)
%
% Revision 1.3  93/05/17  17:06:29  starflt
% Added acknowledgements section (suggested by tompalka)
% 
% Revision 1.2  92/04/22  13:13:13  epeisach
% Fixes for 1991 course 6 requirements
% Phrase "and to grant others the right to do so" has been added to 
% permission clause
% Second copy of abstract is not counted as separate pages so numbering works
% out
% 
% Revision 1.1  92/04/22  13:08:20  epeisach

% NOTE:
% These templates make an effort to conform to the MIT Thesis specifications,
% however the specifications can change. We recommend that you verify the
% layout of your title page with your thesis advisor and/or the MIT 
% Libraries before printing your final copy.
%\title{Machine Learning and Cost Analysis for Risk Management in Geothermal Exploration and Production}
\title{Exploration and Production Risk Mitigation for Geothermal Adoption in the Energy Transition}

\author{Robert Chadwick Holmes}
% If you wish to list your previous degrees on the cover page, use the 
% previous degrees command:
%       \prevdegrees{A.A., Harvard University (1985)}
% You can use the \\ command to list multiple previous degrees
%       \prevdegrees{B.S., University of California (1978) \\
%                    S.M., Massachusetts Institute of Technology (1981)}
\prevdegrees{B.S., Duke University (2000) \\
            M.A., Columbia University (2004) \\
            M.Ph., Columbia University (2006) \\
            Ph.D., Columbia University (2009)}
\department{System Design \& Management Program}

% If the thesis is for two degrees simultaneously, list them both
% separated by \and like this:
% \degree{Doctor of Philosophy \and Master of Science}
\degree{Master of Science in Engineering and Management}

% As of the 2007-08 academic year, valid degree months are September, 
% February, or June.  The default is June.
\degreemonth{September}
\degreeyear{2021}
\thesisdate{August 6, 2021}

%% By default, the thesis will be copyrighted to MIT.  If you need to copyright
%% the thesis to yourself, just specify the `vi' documentclass option.  If for
%% some reason you want to exactly specify the copyright notice text, you can
%% use the \copyrightnoticetext command.  
%\copyrightnoticetext{\copyright IBM, 1990.  Do not open till Xmas.}

% If there is more than one supervisor, use the \supervisor command
% once for each.
\supervisor{Aim\'e Fournier}{Research Scientist, Earth and Planetary Sciences}
\supervisor{Bryan R. Moser}{Academic Director, System Design \& Management Program}

% This is the department committee chairman, not the thesis committee
% chairman.  You should replace this with your Department's Committee
% Chairman.
\chairman{Joan Rubin}{Executive Director, System Design \& Management Program}

% Make the titlepage based on the above information.  If you need
% something special and can't use the standard form, you can specify
% the exact text of the titlepage yourself.  Put it in a titlepage
% environment and leave blank lines where you want vertical space.
% The spaces will be adjusted to fill the entire page.  The dotted
% lines for the signatures are made with the \signature command.
\maketitle

% The abstractpage environment sets up everything on the page except
% the text itself.  The title and other header material are put at the
% top of the page, and the supervisors are listed at the bottom.  A
% new page is begun both before and after.  Of course, an abstract may
% be more than one page itself.  If you need more control over the
% format of the page, you can use the abstract environment, which puts
% the word "Abstract" at the beginning and single spaces its text.

%% You can either \input (*not* \include) your abstract file, or you can put
%% the text of the abstract directly between the \begin{abstractpage} and
%% \end{abstractpage} commands.

% First copy: start a new page, and save the page number.
\cleardoublepage
% Uncomment the next line if you do NOT want a page number on your
% abstract and acknowledgments pages.
% \pagestyle{empty}
\setcounter{savepage}{\thepage}
\begin{abstractpage}
% $Log: abstract.tex,v $
% Revision 1.1  93/05/14  14:56:25  starflt
% Initial revision
% 
% Revision 1.1  90/05/04  10:41:01  lwvanels
% Initial revision
% 
%
%% The text of your abstract and nothing else (other than comments) goes here.
%% It will be single-spaced and the rest of the text that is supposed to go on
%% the abstract page will be generated by the abstractpage environment.  This
%% file should be \input (not \include 'd) from cover.tex.
%Geothermal provides a continuous, low greenhouse-gas emissions source of energy with enormous potential in the United States, both singularly or as part of a renewable energy portfolio. Although a small contributor to the current national energy grid, geothermal capture for generating electricity dates back nearly a century for natural hydrothermal systems. More recently, technologies at various readiness levels give the promise of geothermal access using enhanced geothermal systems (EGS), which provide engineered solutions for subsurface fluid circulation to tap into thermal reservoirs in a wider variety of locations. Nevertheless, the risk of high costs associated with exploration and production remain a hurdle to broader adoption of geothermal as part of a diverse commercial energy mix.

%In this thesis, risk-mitigation strategies for geothermal exploration and production target two separate aspects of the system lifecycle. The first considers how data collected for interrelated earth systems can indicate geothermal potential at the play and prospect scale. Analytical workflows integrating geologic and geophysical data are used to estimate the subsurface geothermal gradient, with quantitative uncertainty estimates associated with the data inputs, the modeling approach, and the size of the solution space. These uncertainty estimates provide a measure of risk, as well as decision tools for investments in additional data-gathering activities before the first well is drilled. The second focus looks at flexibility in engineering design with real options for expanding an existing power facility with geothermal. Specifically, key uncertainties are defined and integrated into a cost-modeling approach that uses decision rules to define an ensemble of possible outcomes. Tailoring the model and decision rules to the potential field and location of interest allows for a rapid but thorough test of project feasibility and the selection of build-out alternatives that limit downside risk and capture upside potential. In total, the learnings from these investigations provide insights into how geothermal can be a commercially viable, low-carbon option as energy companies navigate the ongoing energy transition.
Geothermal provides a continuous, low-emissions source of energy with enormous potential in the United States, both singularly or as part of a broader energy mix. Although a small contributor to the current national energy grid, geothermal electricity generation dates back nearly a century for natural hydrothermal systems. More recently, enhanced geothermal systems (EGS) promise a broader reach with engineered solutions for extracting subsurface heat from a wider variety of locations. The potential synergy between the oil \& gas and geothermal offers an opportunity for building a lower-carbon energy portfolio that requires compatible skills and expertise. Nevertheless, the risks involved at multiple stages of the geothermal field lifecycle remain a hurdle to adoption of geothermal.

In this thesis, risk-mitigation strategies for geothermal target two phases of the lifecycle: exploration and production. The first strategy uses a diverse set of measurements spanning multiple interrelated earth systems to collectively determine geothermal potential at the play scale. Analytical workflows integrate geologic and geophysical data to estimate subsurface geothermal gradient, with quantitative uncertainty estimates associated with the measurements, the models, and the solution space. These uncertainty estimates provide a measure of risk, as well as decision tools for investments in additional data-gathering activities before the first well is drilled. The second strategy applies flexibility in engineering design to a hypothetical EGS expansion of an existing power facility. Specifically, key uncertainties are integrated into a cost model with operational decision rules to create an ensemble of possible outcomes. Tailoring the model and decision rules to a particular facility concept allows for a rapid feasibility testing and optimization of project actions that limit downside risk while capturing upside potential. Both of these strategies use uncertainty characterization to reduce the threat of high-consequence geothermal risks. And by including them in a broader risk management approach, oil \& gas companies can make data-driven decisions on investing in geothermal during the energy transition.
\end{abstractpage}

% Additional copy: start a new page, and reset the page number.  This way,
% the second copy of the abstract is not counted as separate pages.
% Uncomment the next 6 lines if you need two copies of the abstract
% page.
% \setcounter{page}{\thesavepage}
% \begin{abstractpage}
% % $Log: abstract.tex,v $
% Revision 1.1  93/05/14  14:56:25  starflt
% Initial revision
% 
% Revision 1.1  90/05/04  10:41:01  lwvanels
% Initial revision
% 
%
%% The text of your abstract and nothing else (other than comments) goes here.
%% It will be single-spaced and the rest of the text that is supposed to go on
%% the abstract page will be generated by the abstractpage environment.  This
%% file should be \input (not \include 'd) from cover.tex.
%Geothermal provides a continuous, low greenhouse-gas emissions source of energy with enormous potential in the United States, both singularly or as part of a renewable energy portfolio. Although a small contributor to the current national energy grid, geothermal capture for generating electricity dates back nearly a century for natural hydrothermal systems. More recently, technologies at various readiness levels give the promise of geothermal access using enhanced geothermal systems (EGS), which provide engineered solutions for subsurface fluid circulation to tap into thermal reservoirs in a wider variety of locations. Nevertheless, the risk of high costs associated with exploration and production remain a hurdle to broader adoption of geothermal as part of a diverse commercial energy mix.

%In this thesis, risk-mitigation strategies for geothermal exploration and production target two separate aspects of the system lifecycle. The first considers how data collected for interrelated earth systems can indicate geothermal potential at the play and prospect scale. Analytical workflows integrating geologic and geophysical data are used to estimate the subsurface geothermal gradient, with quantitative uncertainty estimates associated with the data inputs, the modeling approach, and the size of the solution space. These uncertainty estimates provide a measure of risk, as well as decision tools for investments in additional data-gathering activities before the first well is drilled. The second focus looks at flexibility in engineering design with real options for expanding an existing power facility with geothermal. Specifically, key uncertainties are defined and integrated into a cost-modeling approach that uses decision rules to define an ensemble of possible outcomes. Tailoring the model and decision rules to the potential field and location of interest allows for a rapid but thorough test of project feasibility and the selection of build-out alternatives that limit downside risk and capture upside potential. In total, the learnings from these investigations provide insights into how geothermal can be a commercially viable, low-carbon option as energy companies navigate the ongoing energy transition.
Geothermal provides a continuous, low-emissions source of energy with enormous potential in the United States, both singularly or as part of a broader energy mix. Although a small contributor to the current national energy grid, geothermal electricity generation dates back nearly a century for natural hydrothermal systems. More recently, enhanced geothermal systems (EGS) promise a broader reach with engineered solutions for extracting subsurface heat from a wider variety of locations. The potential synergy between the oil \& gas and geothermal offers an opportunity for building a lower-carbon energy portfolio that requires compatible skills and expertise. Nevertheless, the risks involved at multiple stages of the geothermal field lifecycle remain a hurdle to adoption of geothermal.

In this thesis, risk-mitigation strategies for geothermal target two phases of the lifecycle: exploration and production. The first strategy uses a diverse set of measurements spanning multiple interrelated earth systems to collectively determine geothermal potential at the play scale. Analytical workflows integrate geologic and geophysical data to estimate subsurface geothermal gradient, with quantitative uncertainty estimates associated with the measurements, the models, and the solution space. These uncertainty estimates provide a measure of risk, as well as decision tools for investments in additional data-gathering activities before the first well is drilled. The second strategy applies flexibility in engineering design to a hypothetical EGS expansion of an existing power facility. Specifically, key uncertainties are integrated into a cost model with operational decision rules to create an ensemble of possible outcomes. Tailoring the model and decision rules to a particular facility concept allows for a rapid feasibility testing and optimization of project actions that limit downside risk while capturing upside potential. Both of these strategies use uncertainty characterization to reduce the threat of high-consequence geothermal risks. And by including them in a broader risk management approach, oil \& gas companies can make data-driven decisions on investing in geothermal during the energy transition.
% \end{abstractpage}

\cleardoublepage

\section*{Acknowledgments}

The past year will be remembered for the impact a global pandemic had on society at large. This thesis is a product of that time. Those mentioned below played a significant role in keeping things on track in spite of the many months of mask-wearing, virus-testing, quarantining, remote classes and conversations, and eventual vaccination. Some I have even met in person, although not all. That is part of the legacy of this most unusual and memorable year.

First and foremost, I want to thank my advisor, Aim\'e Fournier for his good humor, guidance, and willingness to advise under remote conditions. Aim\'e showed early interest in the thesis before a proposal was even drafted, and his perspective and ideas helped shape what it became. I greatly appreciate his time, input, and willingness to take a chance on a stranger with only one year to produce results.

I wish to thank the System Design \& Management program for an intense and incredible educational experience. Remote learning is nothing new to the program, and it was obvious from the quality of instruction they provided. Special thanks to my thesis reader and instructor, Bryan Moser, whose passion for learning and teaching is contagious. Bryan once commented that ``research is a lifestyle,” which rang true in his lectures for the SDM foundations courses, his agent-based modeling course, his Global Teamwork Lab meetings, his IEEE World Forum session, and basically any other time I saw him in action. Thanks also to Joan Rubin for leading the program and supporting each of us throughout the year. On a personal level, Joan kept an open door to communication by email, by phone, and eventually a handful of in-person meetings in support of my thesis journey.

A separate and heartfelt thank you goes to Elizabeth Baker for taking the giant leap from being my teaching assistant in a class of eighty-some students to becoming a very dear friend. Our time spent both online and in-person together made the highs and lows of thesis writing so worth it.

Thanks also to the MIT Energy Initiative for welcoming me among your ranks. Thanks especially to Diane Rigos for the regular check-ins, kind offers of help, and a couple of in-person meals earlier this summer that helped bring the Chevron students together for brief but memorable moments in time.

The Earth Resources Lab was one of my first connections to MIT thanks to their brief visit to Houston in early Spring 2020 (pre-Covid). Thanks go to Laurent Demanet, Director of ERL, for responding to my trial balloon emails and connecting me with both Aim\'e Fournier and Mike Fehler. Mike in turn connected me with Steve Brown, Bill Rodi, Sven Treitel, Connor Smith, and the rest of the Great Basin Machine Learning project team under Jim Faulds. I feel honored to have met and conversed with so many brilliant people whom I now consider friends. Perhaps unsurprisingly, our group conversations were a strong influence on the machine learning investigation in this thesis.

I wish also to thank my company sponsor, Chevron, for taking a chance on me. I specifically want to thank those who believed in my potential and supported my year-long absence to pursue this degree: Sebastien Bombarde, John Moore, and Janet Yun for being my champions; Mason Edwards, Kenn Ehman, Kellen Gunderson, Ash Harris, Fabien Laugier, Rhonda Welch, and Khryste Wright for keeping up with me during the year away; Brendan Horton for being my Digital Scholar mentor, and so many others in the greater Chevron community. On the program side, special thanks go to Shana Bolen and Margery Connor for a year of conversations, encouragement, and a lot of behind-the-scenes efforts that I may not have been fully aware of but certainly appreciate very much.

Being a graduate student at MIT would be daunting in a normal year, but with a pandemic raging and real-life connections to faculty, staff, and fellow students reduced to a laptop screen and webcam, I have the 15 other Chevron scholars to thank for making the year a truly positive and life-changing experience. I feel so honored to have met such a diverse and wonderful group, and I look forward to many years of friendship and collaboration to come: Robert Andrais, Louis Catalan, Gloria Bahl Chambi, Christian Dowell, Matthew Hernandez, Matthew Kieke, Hemant Kumar, Alessandro Lucioli, Elias Machado, Monthep Parimontonsakul, Allison Polly, Kelsey Prestidge, Bagdat Toleubay, John Ward, and my thesis buddy, Surge Yemets.

Last but not least, thank you to my wonderful husband of nearly 8 years, Hans, for putting up with my absence for a full 12 months. You bravely held down the fort on your own, caring for two senior dachshunds and a puppy while balancing a full-time job, commitments to the community, extreme Texas heat, and even the winter storm blackouts that make a cameo in Chapter 1. I dedicate this thesis to you in honor of the impact it had on your life as well as my own.

%%%%%%%%%%%%%%%%%%%%%%%%%%%%%%%%%%%%%%%%%%%%%%%%%%%%%%%%%%%%%%%%%%%%%%
% -*-latex-*-

% Some departments (e.g. 5) require an additional signature page.  See
% signature.tex for more information and uncomment the following line if
% applicable.
% \include{signature}
\pagestyle{plain}
  % -*- Mode:TeX -*-
%% This file simply contains the commands that actually generate the table of
%% contents and lists of figures and tables.  You can omit any or all of
%% these files by simply taking out the appropriate command.  For more
%% information on these files, see appendix C.3.3 of the LaTeX manual. 

{\hypersetup{linkcolor=black}\tableofcontents}
\cleardoublepage
\addcontentsline{toc}{chapter}{\listfigurename}
{\hypersetup{linkcolor=black}\listoffigures}
\clearpage
\addcontentsline{toc}{chapter}{\listtablename}
{\hypersetup{linkcolor=black}\listoftables}
\clearpage

\printnoidxglossary[type=acronym, title=List of Acronyms, toctitle=List of Acronyms]
\printacronyms
\cleardoublepage
%% This is an example first chapter.  You should put chapter/appendix that you
%% write into a separate file, and add a line \include{yourfilename} to
%% main.tex, where `yourfilename.tex' is the name of the chapter/appendix file.
%% You can process specific files by typing their names in at the 
%% \files=
%% prompt when you run the file main.tex through LaTeX.
\chapter{Introduction}\label{ch1:intro}
The pursuit of energy has shaped the story of mankind from the very beginning. And while the image of ancient humans using fires for warmth, protection, and meal preparation is an archetype of our distant past, modern human needs remain much the same. Lighting to extend day into night, heating and cooling for comfort in our homes, cooking of the food we eat, access to the advanced technologies of our time – these all require energy from one source or another.  Choices abound, from animal and plant-based fuels, to buried resources like various forms of hydrocarbons, to alternatives like solar, wind, hydro, nuclear, and geothermal. How these sources and resources are balanced can shape the growth of societies on the geopolitical stage, as well as grander scale impacts like the future of a habitable Earth and the perseverance and resilience of mankind.

This thesis examines how uncertainty characterization and risk mitigation can increase the role of one source, geothermal, in addressing the ever-growing energy needs in a viable way. This chapter reflects on the extent of those needs and the conditions that may uniquely support an increased focus on geothermal in the near-term. Opportunities and challenges associated with geothermal also lay the foundation for research questions motivating the remainder of this body of work.

\section{Energy Trends}\label{ch1:trends}
The \acrlong{eia} (\acrshort{eia}) publishes annual forecasts on U.S. energy generation and consumption in the \acrlong{aeo} report. Based on the 2020 reference case, the \acrshort{aeo} model predicts a 70\% increase in U.S. energy consumption by 2050 driven primarily by the industrial and power sectors \citep{us_energy_information_administration_annual_2021}. Electricity generation also grows by one third, driven primarily by renewables and natural gas as coal, nuclear, and oil see reductions (Figure \ref{fig:eia_2021_projections}). These predictions are offered with the caveat of much greater uncertainty due to the impact of the COVID-19 pandemic, although the \acrshort{eia} suggests a return to normal will occur by 2025 and broader, decadal trends will be unchanged \citep{us_energy_information_administration_annual_2021}. International forecasts show similar growth in consumption and production, but traditional sources of energy like coal and natural gas also increase in capacity to meet the needs of India, China, and other rapidly developing nations \citep{us_energy_information_administration_international_2020}. 
 
\begin{figure}[htp]
\centering
\includegraphics[scale=0.6]{Figure001-EIA_projections}
\caption[U.S. EIA projections based on the AEO2021 reference case]{U.S. EIA projections of (Left) U.S. electricity generation by sector and (Right) individual contributions by renewable type based on the AEO2021 reference case \protect\citep{us_energy_information_administration_annual_2021}}
\label{fig:eia_2021_projections}
\end{figure}

Lazard Asset Management breaks renewables down by \acrlong{lcoe} (\acrshort{lcoe}) in U.S. dollars/\acrshort{mwh}, where \acrshort{lcoe} is the estimated lifetime average net cost per unit energy of an electricity generating plant. In their 2020 analysis, intermittent energy sources like wind and utility-scale solar are already cost-competitive with fossil fuel-derived sources \citep{lazard_lazards_2020}. Geothermal, an “always on” source of power, ranges from \$59-\$101/\acrshort{mwh}  \acrshort{lcoe}, making it second-tier in cost competitiveness but comparable to community and rooftop solar installations \citep{lazard_lazards_2020}. Overall, the transition in energy sources away from fossil fuel dominance is in progress, and the demand for energy in support of population growth and country development will remain driver over the next 30 years.

%section~\ref{ch1:sec}.

\section{Upstream Commercial Pressures}\label{ch1:upstream}


\section{Net Zero Ambitions}\label{ch1:netzero}


\section{Geothermal Energy}\label{ch1:geothermal}

\section{Research Questions}\label{ch1:researchqs}


\section{Section sample 2}\label{ch1:sec}

%\footnote{Here is a sample footnote referencing figures~\ref{arm:fig1}
%and~\ref{arm:fig2}.}  


% This is an example of how you would use tgrind to include an example
% of source code; it is commented out in this template since the code
% example file does not exist. To use it, you need to remove the '%' on the
% beginning of the line, and insert your own information in the call.
%
%\tagrind[htbp]{code/pmn.s.tex}{Code sample}{opt:pmn}

%\subsection{Subsection with list}
%\begin{enumerate}
%  \item Item 1.
%  \item Item 2.
%  \item Item 3.
%\end{enumerate}


% This is an example of how you would use tgrind to include an example
% of source code; it is commented out in this template since the code
% example file does not exist.  To use it, you need to remove the '%' on the
% beginning of the line, and insert your own information in the call.
%
%\tgrind[htbp]{code/be.s.tex}{Block Exponent}{opt:be}

%\subsection{Another subsection sample}

%This is done by using some combination of
%\begin{eqnarray*}
%a_i & = & a_j + a_k \\
%a_i & = & 2a_j + a_k \\
%a_i & = & 4a_j + a_k \\
%a_i & = & 8a_j + a_k \\
%a_i & = & a_j - a_k \\
%a_i & = & a_j \ll m \mbox{shift}
%\end{eqnarray*}
%instead of the multiplication.  For example, you could use:
%\begin{eqnarray*}
%r & = & 4s + s\\
%r & = & r + r
%\end{eqnarray*}
%Or by xx:
%\begin{eqnarray*}
%t & = & 2s + s \\
%r & = & 2t + s \\
%r & = & 8r + t
%\end{eqnarray*}

%% This is an example first chapter.  You should put chapter/appendix that you
%% write into a separate file, and add a line \include{yourfilename} to
%% main.tex, where `yourfilename.tex' is the name of the chapter/appendix file.
%% You can process specific files by typing their names in at the 
%% \files=
%% prompt when you run the file main.tex through LaTeX.
\chapter{Background}\label{ch2:background}

Cras pharetra ligula nec lectus bibendum, euismod mattis purus cursus. Nullam ut mi molestie purus ultricies lacinia. Phasellus sed orci ac lacus convallis vestibulum. Quisque id nulla ut ipsum finibus vehicula. Curabitur scelerisque erat lobortis, dapibus purus eget, faucibus sapien. Nam enim leo, faucibus id ante sed, fringilla luctus eros. Morbi vulputate, purus at commodo aliquet, turpis dolor sollicitudin libero, id vehicula risus dui sit amet nulla. Sed auctor efficitur urna. Praesent sagittis tellus ac velit vestibulum dignissim. Vivamus justo enim, pellentesque eu posuere id, mattis vitae felis. Aliquam id tincidunt diam. Class aptent taciti sociosqu ad litora torquent per conubia nostra, per inceptos himenaeos. Pellentesque habitant morbi tristique senectus et netus et malesuada fames ac turpis egestas.



\section{Section sample 2}\label{ch1:sec}



%section~\ref{ch1:sec}.

%\footnote{Here is a sample footnote referencing figures~\ref{arm:fig1}
%and~\ref{arm:fig2}.}  


% This is an example of how you would use tgrind to include an example
% of source code; it is commented out in this template since the code
% example file does not exist. To use it, you need to remove the '%' on the
% beginning of the line, and insert your own information in the call.
%
%\tagrind[htbp]{code/pmn.s.tex}{Code sample}{opt:pmn}

%\subsection{Subsection with list}
%\begin{enumerate}
%  \item Item 1.
%  \item Item 2.
%  \item Item 3.
%\end{enumerate}


% This is an example of how you would use tgrind to include an example
% of source code; it is commented out in this template since the code
% example file does not exist.  To use it, you need to remove the '%' on the
% beginning of the line, and insert your own information in the call.
%
%\tgrind[htbp]{code/be.s.tex}{Block Exponent}{opt:be}

%\subsection{Another subsection sample}

%This is done by using some combination of
%\begin{eqnarray*}
%a_i & = & a_j + a_k \\
%a_i & = & 2a_j + a_k \\
%a_i & = & 4a_j + a_k \\
%a_i & = & 8a_j + a_k \\
%a_i & = & a_j - a_k \\
%a_i & = & a_j \ll m \mbox{shift}
%\end{eqnarray*}
%instead of the multiplication.  For example, you could use:
%\begin{eqnarray*}
%r & = & 4s + s\\
%r & = & r + r
%\end{eqnarray*}
%Or by xx:
%\begin{eqnarray*}
%t & = & 2s + s \\
%r & = & 2t + s \\
%r & = & 8r + t
%\end{eqnarray*}

\chapter{Exploration Analytics Methodology}\label{ch3:expl_methods}

\section{Exploration Data Sources}\label{ch3:expl_data_src}

This investigation brings together a total of twenty-five (25) data sets covering the southwest NM study area. Data were collected from previously published works, open-access databases, or derived from those original sources as secondary products. The form of the data varies between pre-gridded raster files, point data sets with repeat or overlapping measurements, non-overlapping point sets, and line data. Previous researchers created raster files or raster-ready gridded data for nine of the features. Four are generated by running procedures on one of the existing rasters. The remaining layers were created from polylines (3), overlapping points (4), and non-overlapping points (2). Although complex interactions between earth systems should be expected, these layers represent the independent variables for analysis purposes. Section XXX details how evaluating collinearities between features allows for pre-screening before modeling, and further analysis of feature importances helps reduce this composite data set to a smaller subset for simpler prediction models.

\begin{table}[htp]
\centering
\includegraphics[scale=1]{Table-Features.png}
\caption[Features considered in this the exploration analytics study]{List of data sets considered in this study. Data type, provider, and source location are listed. Numbered features are treated as independent variables. 'D' indicates the dependent variable.}
\label{tab:features}
\end{table}

As discussed in Section \ref{ch2:sysfund}, geothermal systems require permeability, heat, subsurface fluids, a trapping mechanism, and recharge. Following the slightly more simplified PFA assumptions of \citeauthor{bielicki_hydrogeolgic_2015} (\citeyear{bielicki_hydrogeolgic_2015}), an explorationist will want to quickly identify where heat, permeability, and fluids together define a favorable setting for a geothermal prospect. The prepared independent data inputs collectively address all three elements as noted in Table \ref{tab:features}. Rather than defining a dependent (predicted) variable that describes a total favorability score, this thesis focuses on a proof of concept prediction for a measurable quantity addressing just the heat risk element: geothermal gradient. This choice was made because a) geothermal gradient provides a direct proxy for accessible heat content, b) gradient point data is available from suitable compilations of well measurements collected across the study area, and c) for EGS applications, the only risk element that must be naturally present is heat. Heat flow might be a reasonable alternative dependent variable, however point values for heat flow in the available well database were derived directly from geothermal gradient values. Geothermometer measurements also suggest resource temperatures, but the uncertainty in fluid pathways leading to the sample location means these values suffer from less spatial and depth certainty than geothermal gradient.

Regarding the remaining two risk elements: direct measurements of permeability (i.e., from downhole logs or core analysis) or fluids (e.g., flow rate from well tests) can be separately predicted using the same methods described in this study. A final favorability score, which is less straight-forward to calibrate for model validation and verification, could potentially be derived from the combined predictions as done in PFA risk assessments. This suggestion is outside of the scope of this thesis and thus appears in the list of future work opportunities (see Chapter 9).

\section{Exploration Data Preparation}

In order to experiment with a variety of machine learning methods, all input data sets first need to be transformed into fully-complete \acrlong{gis} (\acrshort{gis}) layers such that any point on the map of the study area has a corresponding set of 25 independent feature values. Steps taken to condition, process, and otherwise prepare each layer are outlined later in this chapter. As a preface, the following section reviews several key algorithms and concepts applied to one or more of the layers for clarity and reproducibility.

\subsection{Data Preparation Algorithms}

\subsubsection{Extents}

Large data sets imported into ArcGIS or Python for feature preparation required cropping to the southwest New Mexico study area. Two polygons were used for this purpose:

\begin{itemize}
\item Extent Polygon: this is a simple rectangular polygon capturing the broader southwestern NM region. It is defined by the following corner points in degrees N Latitude and degrees E Longitude: \\ (-31.3, -109.1), (31.3, -105.9), (35.4, -105.9), (31.3, -109.1)
\item \acrlong{aoi} (\acrshort{aoi}): this polygon appears in most map figures in this thesis and is the perimeter outlining the nine counties in Southwest New Mexico: Cibola, Valencia, Catron, Socorro, Grant, Sierra, Luna, Dona Ana, and Hidalgo.
\end{itemize}

\subsubsection{Fishnet Points}\label{ssn:fishnet}

\subsubsection{Simple Kriging}\label{ssn:kriging}

\subsubsection{Empirical Bayes Kriging}\label{ssn:ebk}

\subsubsection{Splines}

\subsubsection{Topo to Raster?}

\subsubsection{\acrlong{kde} (\acrshort{kde})}\label{ssn:kde}

\subsection{Data Layers}

\subsubsection{Average Air Temperature}

The University of Oregon PRISM Climate Group hosts regularly-updated spatial data sets of climate-related observations captured from different monitoring networks, including 30-year normals that describe average monthly or annual conditions \citep{daly_physiographically_2008, prism_prism_2021}. 800 m or 4 km resolution grids can be accessed directly from the website. The 800 m resolution air temperature grid was downloaded and imported into ArcGIS, then cropped using the Extent Polygon (Figure \ref{fig:feat_airtemp}). The layer required no further processing.

\begin{figure}[!htp]
\centering
\includegraphics[scale=.50]{Figure-AvgAirTemp}
\caption[Average air temperature data layer]{Average air temperature data layer. Units are in degrees Celsius. Data retrieved from \protect\citep{prism_prism_2021}.}
\label{fig:feat_airtemp}
\end{figure}

\subsubsection{Average Precipitation}

The University of Oregon PRISM Climate Group also compiles 30-year normals for average precipitation \citep{daly_physiographically_2008, prism_prism_2021}. The 800 m resolution precipitation grid was downloaded and imported into ArcGIS, then cropped to the Extent Polygon boundaries (Figure \ref{fig:feat_precip}). The layer required no further processing.

\begin{figure}[!htp]
\centering
\includegraphics[scale=.50]{Figure-AvgPrecip}
\caption[Average precipitation data layer]{Average precipitation data layer in 800 m resolution. Units are in millimeters. Data retrieved from \protect\citep{prism_prism_2021}.}
\label{fig:feat_precip}
\end{figure}

\subsubsection{Basement Depth}

Following the procedure of \citeauthor{pepin_new_2018} (\citeyear{pepin_new_2018}), the basement elevation raster generated by \citeauthor{bielicki_hydrogeolgic_2015} (\citeyear{bielicki_hydrogeolgic_2015}) was downloaded, imported into ArcGIS, and processed to calculate depths. Specifically, a unit conversion from feet to meters was applied. Then, values were extracted on the point fishnet (see \ref{ssn:fishnet}), which highlighted missing data patches in the data. The ArcGIS \textit{Kriging} function interpolated values across these patches using the Ordinary method with Spherical semivariogram, a lag size of 0.096969 automatically determined by ArcGIS, and a variable search radius with a 4-point requirement. Basement depths were then calculated by subtracting the interpolated elevation layer from the surface topography (DEM) layer. However, the higher resolution of the DEM layer caused an imprint of detailed surface geomorphologies to appear on the calculated basement depth layer. To correct for this, the DEM layer was low pass filtered using the ArcGIS \textit{Filter} method, which averages a 3x3 neighborhood around each point in the data set. The final basement elevation layer (Figugre \ref{fig:feat_basementdepth}) was generated from the difference between the low-pass filtered DEM and the kriged basement depth.

\begin{figure}[h!]
\centering
\includegraphics[scale=.50]{Figure-BasementDepth}
\caption[Basement depth data layer]{Basement depth data layer. Units are in meters. Layer based on basement elevation raster from \protect\citep{bielicki_hydrogeolgic_2015}.}
\label{fig:feat_basementdepth}
\end{figure}

\subsubsection{Boron Concentration}

Measurements of boron concentration were originally assembled by \citeauthor{bielicki_hydrogeolgic_2015} (\citeyear{bielicki_hydrogeolgic_2015}) from multiple sources ranging from USGS records to student dissertations. These data were downloaded from the OpenEI submission \citep{kelley_geothermal_2015} and merged together using ArcGIS and Python to create a single dataframe of 5686 measurements, all within the broader Extent Polygon bounds to avoid surface creation edge effects within the tighter AOI. The inconsistent spatial distribution of the data and sometimes significant variation among overlapping values from different measurement years created a unique challenge for making a representative GIS layer to use for analysis. An initial attempt to fit and interpolate the data using tuned Gaussian Process models created feature layers with too much local structure and little character away from the input data points. The ArcGIS \textit{Empirical Bayes Kriging} routine was selected instead due to its unique characteristics. For the final layer, EBK was applied with the Empirical data transformation type, a maximum of 100 points in each local model, 100 simulated semivariograms with K-Bessel model type, and a standard circular search neighborhood with a radius of 1.1957 (auto-generated), minimum of 10 neighbors, and maximum of 15 neighbors. The output grid cell size was set to 0.01 degrees. Of important note: the calculation option to include all coincident data was selected, so all overlapping measurements were considered in generating the final layer (Figure \ref{fig:feat_boron}).

\begin{figure}[!htp]
\centering
\includegraphics[scale=.50]{Figure-Boron}
\caption[Boron concentration data layer]{Boron concentration data layer. Units are in ppm or mg/L. Black dots indicate sample locations in complete data set from \protect\citep{bielicki_hydrogeolgic_2015}.}
\label{fig:feat_boron}
\end{figure}

\subsubsection{Crustal Thickness}

In the absence of a more recent seismic study constraining variations in crustal thickness across the study area, the 2.5D regional map published by \citeauthor{keller_comparative_1991} (\citeyear{keller_comparative_1991}) was to construct the crustal thickness feature layer. Similar to the procedure described in \citep{pepin_new_2018}, the Keller map was georeferenced in ArcGIS, and thickness contours were manually digitized as polylines. These polylines continued slightly beyond the AOI boundary to ensure proper constraints for surface creation without artifacts near the AOI edges. The ArcGIS function \textit{Feature to 3D by Attribute} converted the polylines into 3D contours, and \textit{Topo to Raster} interpolated the contours into a continuous final grid (Figure \ref{fig:feat_crust}). Since the Keller map was derived from low-resolution seismic lines from the 1960s-1980s, the result is a very low frequency approximation for crustal thickness variations associated with the CP and RGR. As such, \textit{Topo to Raster} a larger cell size of 0.025 degrees was used. Other parameters choices include: margin in the cells of 20, smallest z value for interpolation of 25, largest z value for the interpolation of 55, Enforce selection for drainage enforcement, and maximum iterations of 20.

\begin{figure}[!htp]
\centering
\includegraphics[scale=.50]{Figure-CrustalThickness}
\caption[Crustal thickness data layer]{Crustal thickness data layer. Units are in kilometers. Black lines refer to contours digitized from \protect\citep{keller_comparative_1991}.}
\label{fig:feat_crust}
\end{figure}

\subsubsection{Drainage Density}

Drainage polyline data comes from the \citep{bielicki_hydrogeolgic_2015} PFA OpenEI submission \citep{kelley_geothermal_2015}. The data were downloaded and imported into ArcGIS, then compared to the DEM later for quality control. A couple of methods were attempted in order to transform this feature into a continuous-valued layer with full map coverage. First, the polylines were converted to points with 500 m sampling. This point set was loaded into a Python script, which used a grid search routine to determine the best radius for a Gaussian kernel density operator. Ten-fold cross validation was employed, which splits the data into 10 subsets and interchangeably trains on 9, tests on one to get an average performance score. Based on a calculation of the negative log likelihood, the best radius was found to be 45,600 m. However, when the kernel density operation is applied to the data with this radius, the map shows a central blob of high density, which falls off toward the sides of the survey. With such a high radius, edge effects come into play since no drainage polylines were available outside of the AOI boundary. Furthermore, the conversion of line data to points for this method disregards the spatial relationships of the connected line data. The ArcGIS version of kernel density uses an adapted quartic kernel function described by \citep{silverman_density_2018} to fit a smoothly-curved surface over polylines with a maximum value over the line and sides that taper to zero based on the search radius. Default search radius values come from a bandwidth calculation, also based on Silverman book \citep{esri_how_2021,silverman_density_2018}. Densities are calculated as the cumulative sum of the overlapping curve fits for all polylines in a data set. This method operates on the polyline features directly and results in a map with much more drainage variation within the AOI. The final drainage density layer (Figure \ref{fig:feat_drainage}) was generated using this methodology with an output cell size of 0.0025 degrees and an auto-generated search radius of 0.2716.

\begin{figure}[!htp]
\centering
\includegraphics[scale=.50]{Figure-Drainage}
\caption[Drainage density data layer]{Drainage density data layer. Units are in degrees/square degrees. Blue lines show the drainage polyline data set from \protect\citep{bielicki_hydrogeolgic_2015}.}
\label{fig:feat_drainage}
\end{figure}

\subsubsection{Earthquake Density}

Following the procedure outlined by \citeauthor{pepin_new_2018} (\citeyear{pepin_new_2018}), an earthquake data set for southwest New Mexico was created by combining historical earthquake catalogs for 1869-1998 \citep{sanford_earthquake_2002}, 1999-2004 \citep{sanford_earthquake_2006}, and 2005-2009 \citep{pursley_earthquake_2013} with data pulled from the USGS Earthquake catalog \citep{usgs_earthquake_2021} through to January 2021. All events were combined into a single dataframe in Python, and event duplicates were removed. The final catalog, cropped to the Extent Polygon boundary, consists of 2539 events spanning 1962-2020. This point set was loaded into a KDE Python script, which used a grid search routine to determine the best radius for a Gaussian kernel density operator. Ten-fold cross validation was employed, which splits the data into 10 subsets and then interchangeably trains on 9 and tests on one to get an average performance score. The maximum negative log likelihood indicates a best radius value of 11,600 m (Figure \ref{fig:EQ_cv}). 

\begin{figure}[!htp]
\centering
\includegraphics[scale=.50]{templates/images/Figure-Earthquake_kde_gridsearchcv_result.png}
\caption[Earthquake density parameter tuning]{Cross-validation results for earthquake KDE. Red dashed line indicates maximum negative log likelihood value identifying the best choice for kernel radius.}
\label{fig:EQ_cv}
\end{figure}

KDE values calculated at each fishnet point location were loaded into ArcGIS, and \textit{Kriging} was used to generate a final surface for plotting purposes. textit{Kriging} parameters included Ordinary kriging method, Spherical semivariogram model, lag size of 1e-6, and a variable search radius with 12-point requirement. The final earthquake density map (Figure \ref{fig:feat_EQ_density}) appears similar to Figure 3.10 in (Pepin, 2018). 

\begin{figure}[!htp]
\centering
\includegraphics[scale=.50]{Figure-EarthquakeDensity}
\caption[Earthquake density data layer]{Earthquake density data layer. Units are in log density (points/\(km^2\)). Block dots indicate earthquake locations.}
\label{fig:feat_EQ_density}
\end{figure}

\subsubsection{Gamma Ray Dose Rate}

Aerial gamma-ray surveys conducted across the United States in the late 1970-1980s allowed for the construction of Potassium concentration (K, in percent K), Uranium concentration (eU, in ppm), and Thorium concentration (eTh in ppm) maps, which relate to mineralogy, lithology, and hence, stratigraphy. These measures collectively define the absorbed dose rate, which can be calculated from the following equation: \( D = 13.2 K + 5.48 eU + 2.72 eTh \) \citep{duval_terrestrial_2005}.

The absorbed dose rate for West Central USA was downloaded from the USGS Open-File Report 2005-1413 site \citep{duval_terrestrial_2005}, loaded into ArcGIS, and cropped to the Extent Polygon bounds. A data gap in the vicinity of the White Sands Missile Range to the southeast of the study area necessitated layer interpolation using kriging. Grid values were extracted using the fishnet of points, then passed through the ArcGIS \textit{Kriging} function for the final layer (Figure \ref{fig:feat_gamma}) with the following parameters: Ordinary kriging method, Spherical semivariogram model, auto-generated lag size of 0.096969, and a variable search radius with a 4-point requirement.

\begin{figure}[!htp]
\centering
\includegraphics[scale=.50]{templates/images/Figure-AbsorbedDoseRate.png}
\caption[Absorbed dose rate data layer]{Absorbed dose rate data layer. Units are in nanoGrays/hour (nGy/hr). Original data from USGS Open-File Report 2005-1413 \protect\citep{duval_terrestrial_2005}}.
\label{fig:feat_gamma}
\end{figure}

\subsubsection{Geodetic Strain Rate}

GPS stations worldwide record local movements in the Earth’s crust. These movements can highlight inflation or subsidence of the surface, fault motions, or plate tectonic activity. The spatial derivative of crustal velocities is termed the strain rate, which gives an indication of the accumulation of strain in an area. More concretely, it defines the speed with which the crust is deforming, and can be treated as a proxy for earthquake potential since slip occurs due to the accumulation of strain \citep{gem_strain_2014}. The Global Strain Rate Model (GSRM) v.2.1 provides a model for strain rate based on over 22,000 measurements from over 18,000 locations around the world \citep{kreemer_geodetic_2014}. The output of this model was downloaded from the University of Nevada Reno Geodetic Laboratory host site \citep{kreemer_corne_global_2020}. GSRM describes elements of the full strain tensor at a resolution 0.1 degrees. The magnitude or second invariant of the strain tensor is defined as \citep{kreemer_geodetic_2014}:

\begin{equation}\label{eq:strainratemagnitude}
\left\lVert\dot{\epsilon}\right\rVert = \sqrt{\dot{\varepsilon}_{\phi\phi}^2+\dot{\varepsilon}_{\theta\theta}^2+2\dot{\varepsilon}_{\phi\theta}^2}
\end{equation}

Due to the size of the model file and complexity of this calculation, the data was first loaded into Python, cropped to the Extent Polygon bounds, and the strain rate magnitude was calculated for each point. These data were then loaded into ArcGIS and gridded using the \textit{Spline} function for a smooth interpolation of the coarser GSRM grid. The final layer (Figure \ref{fig:feat_strain}) was created using \textit{Spline} parameters: Regularized type, weight of 0.1, a 4-point requirement, and an output cell size of 0.025 degrees.

\begin{figure}[!htp]
\centering
\includegraphics[scale=.50]{templates/images/Figure-StrainRate.png}
\caption[Geodetic strain rate data layer]{Geodetic strain rate data layer. Units are in \(10^-9 m/(m*yr)\). Layer is based on data from \protect\citep{kreemer_geodetic_2014}}.
\label{fig:feat_strain}
\end{figure}

\subsubsection{Gravity Anomaly}

Terrain-corrected gravity anomaly data available from the University of Texas El Paso \citep{utep_gravity_2011} were used in both the southwest NM PFA analysis \citep{bielicki_hydrogeolgic_2015} and cluster analysis \citep{pepin_new_2018}. The data layer from \citeauthor{bielicki_hydrogeolgic_2015} was downloaded from their OpenEI submission \citep{kelley_geothermal_2015} and loaded into ArcGIS. The layer (Figure \ref{fig:feat_gravity}) required no further processing.

\begin{figure}[!htp]
\centering
\includegraphics[scale=.50]{templates/images/Figure-GravityAnomaly.png}
\caption[Gravity anomaly data layer]{Gravity anomaly data layer. Units are in milligals (mGal). Raster originally created by \protect\citet{bielicki_hydrogeolgic_2015}.}.
\label{fig:feat_gravity}
\end{figure}

\subsubsection{Gravity Anomaly Gradient}

Gradient of the gravity anomaly was calculated using the ArcGIS Slope function on the final gravity anomaly raster. Parameters used to create the final layer (Figure \ref{fig:feat_gravity_gradient} include: Geodesic method, Z unit of meters, and output measurement of degrees.

\begin{figure}[!htp]
\centering
\includegraphics[scale=.50]{templates/images/Figure-GravityGradient.png}
\caption[Gravity anomaly gradient data layer]{Gravity anomaly gradient data layer. Units are in mGal/degrees.}.
\label{fig:feat_gravity_gradient}
\end{figure}

\subsubsection{Heat Flow}

The recent $0.5^\circ$ x$0.5^\circ$ resolution heat flow model from \citet{lucazeau_analysis_2019} offers coarse coverage across the southwest NM AOI. The model output was downloaded from the supporting information section of the publication page \citep{lucazeau_analysis_2019}, loaded into ArcGIS, and cropped to the Extent Polygon boundaries. After testing several gridding algorithms for a smooth representation of this sparse data, the ArcGIS \textit{Topo to Raster} function produced the best results. The final layer (Figure \ref{fig:feat_heatflow}) parameters include: tolerance 1 of 2.5, tolerance 2 of 100, drainage enforcement set to Enforce, Contour selected for primary type of input data, and output cell size of 0.01. 

\begin{figure}[!htp]
\centering
\includegraphics[scale=.50]{templates/images/Figure-HeatFlow.png}
\caption[Heat flow data layer]{Heat flow data layer. Units are in W/m-K. Original data from \protect\citep{lucazeau_analysis_2019}.}.
\label{fig:feat_heatflow}
\end{figure}

\subsubsection{Lithium Concentration}

Measurements of Lithium concentration were originally assembled by \citet{bielicki_hydrogeolgic_2015} from several sources ranging from USGS records to student dissertations. These data were downloaded from the OpenEI submission \citep{kelley_geothermal_2015} and merged together using ArcGIS and Python to create a single dataframe of 3595 measurements, all within the broader Extent Polygon bounds to avoid surface creation edge effects within the tighter AOI. As described for the Boron concentration data layer, attempts to model Lithium concentration using Gaussian Processes provided unsatisfactory results. Instead, the final layer was generated using the ArcGIS \textit{Empirical Bayes Kriging} routine. Selected parameters include: Empirical data transformation type, a maximum of 100 points in each local model, 100 simulated semivariograms with K-Bessel model type, and a standard circular search neighborhood with a radius of 1.1957 (auto-generated), minimum of 10 neighbors, and maximum of 15 neighbors. The output grid cell size was set to 0.01 degrees. All coincident data was included in the calculation, so any overlapping measurements were considered in generating the final layer (Figure \ref{fig:feat_lithium}).

\begin{figure}[!htp]
\centering
\includegraphics[scale=.50]{templates/images/Figure-Lithium.png}
\caption[Lithium concentration data layer]{Lithium concentration data layer. Units in ppm or mg/L. Black dots indicate sample locations in complete data set from \protect\citep{bielicki_hydrogeolgic_2015}}.
\label{fig:feat_lithium}
\end{figure}

\subsubsection{Magnetic Anomaly}

USGS magnetic anomaly data derived from aerial surveys \citep{bankey_digital_2002} were used in both the southwest NM PFA analysis \citep{bielicki_hydrogeolgic_2015} and cluster analysis \citep{pepin_new_2018}. After downloading the raster from the southwest NM PFA OpenEI submission \citep{kelley_geothermal_2015}, it was loaded into ArcGIS. The layer (Figure \ref{fig:feat_magnetics}) required no further processing.

\begin{figure}[!htp]
\centering
\includegraphics[scale=.50]{templates/images/Figure-MagneticAnomaly.png}
\caption[Magnetic anomaly data layer]{Magnetic anomaly data layer. Units are in nanoteslas (nT). Raster originally created by \protect\citet{bielicki_hydrogeolgic_2015}.}.
\label{fig:feat_magnetics}
\end{figure}

\subsubsection{Magnetic Anomaly Gradient}

The gradient of the magnetic anomaly was calculated using the ArcGIS Slope function on the final magnetic anomaly raster. Parameters used to create the final layer (Figure \ref{fig:feat_magnetic_gradient}) include: Geodesic method, Z unit of meters, and output measurement of degrees.

\begin{figure}[!htp]
\centering
\includegraphics[scale=.50]{templates/images/Figure-MagneticGradient.png}
\caption[Magnetic anomaly gradient data layer]{Magnetic anomaly gradient data layer. Units are in nT/degrees.}.
\label{fig:feat_magnetic_gradient}
\end{figure}

\subsubsection{Quaternary Fault Density}

Faults showing Quaternary displacement were digitized at the 1:24,000 scale by the New Mexico Bureau of Geology and Mineral Resources and provided by NM BGMR to \citet{bielicki_hydrogeolgic_2015} and \citet{pepin_new_2018} in support of their investigations. The associated polyline features were downloaded from the PFA OpenEI submission \citep{kelley_geothermal_2015} and loaded into ArcGIS. As discussed for the Drainage density layer, a Python-based kernel density workflow using extracted points from these polylines failed to produce satisfactory results. Instead, the ArcGIS \textit{Kernel Density} function was applied to create the final layer map (Figure \ref{fig:feat_qfaults}). Selected parameters for this function include an output cell size of 0.0025 degrees and an auto-generated search radius of 0.3672.

\begin{figure}[!htp]
\centering
\includegraphics[scale=.50]{templates/images/Figure-QFaultDensity.png}
\caption[Quaternary fault density data layer]{Quaternary fault density data layer. Units are in degrees/square degress. Black lines show the fault polyline data set archived by \protect\citet{bielicki_hydrogeolgic_2015}.}.
\label{fig:feat_qfaults}
\end{figure}

\subsubsection{Silica Geothermometer Temperature}

Silica concentration data from across the study area were compiled by \citet{bielicki_hydrogeolgic_2015}, and converted to reservoir temperatures using the Fournier chalcedony geothermometer relationship \citep{fournier_chemical_1977}. These data were downloaded from the PFA OpenEI submission \citep{kelley_geothermal_2015} and merged together using ArcGIS and Python to create a single dataframe of 7259 measurements, all within the broader Extent Polygon bounds to avoid surface creation edge effects within the tighter AOI. As described for the Boron concentration data layer, attempts to model Si geothermometer estimates using Gaussian Processes provided unsatisfactory results. Instead, the final layer was generated using the ArcGIS \textit{Empirical Bayes Kriging} routine. Selected parameters include: Empirical data transformation type, a maximum of 100 points in each local model, 100 simulated semivariograms with K-Bessel model type, and a standard circular search neighborhood with a radius of 1.1957 (auto-generated), minimum of 10 neighbors, and maximum of 15 neighbors. The output grid cell size was set to 0.01 degrees. All coincident data was included in the calculation, so any overlapping measurements were considered in generating the final layer (Figure \ref{fig:feat_si_temp}).

\begin{figure}[!htp]
\centering
\includegraphics[scale=.50]{templates/images/Figure-SiTemp.png}
\caption[Silica geothermometer temperature data layer]{Chalcedony geothermometer data layer. Units are in degrees C. Black dots indicate locations where silica concentration was sampled, as collected by \protect\citet{bielicki_hydrogeolgic_2015}}.
\label{fig:feat_si_temp}
\end{figure}

\subsubsection{Water Table Depth}

\citeauthor{bielicki_hydrogeolgic_2015} (\citeyear{bielicki_hydrogeolgic_2015}) mapped the depth to water table using data from the USGS and several additional sources. This raster was downloaded from their OpenEI submission \citep{kelley_geothermal_2015} and imported into ArcGIS. Data gaps between the raster extent and AOI polygon to the south and the east necessitated extrapolation of the layer, so ArcGIS \textit{Empirical Bayes Kriging} was applied to fill in the missing edge values. After some trial-and-error, the chosen parameter values for the final layer (Figure \ref{fig:feat_wtdepth}) include an output cell size of 0.01, Empirical data transformation type, a maximum of 100 points in each local model, 100 simulated semivariograms with Exponential model type, and a standard circular search neighborhood with a radius of 1.2652 (auto-generated), minimum of 10 neighbors, and maximum of 15 neighbors.

\begin{figure}[!htp]
\centering
\includegraphics[scale=.50]{Figure-WTDepth}
\caption[Water table depth data layer]{Water table depth data layer. Units in feet. Adapted from raster created by \protect\citep{bielicki_hydrogeolgic_2015}.}
\label{fig:feat_wtdepth}
\end{figure}
%\chapter{Power Plant Cost Model Preparation}\label{ch4:cm_prep}

\section{EGS Expansion Concept}\label{ch4:cm_concept}
\subsection{Lightning Dock EGS}

Lightning Dock (Section \ref{ch2:lightning_dock}) is presently the only commercial power plant operating in the state of New Mexico. The net generating capacity after its first phase of development was 4 MW in 2013, with the expectation of upgrading to 10 MW in a second development phase that never came to fruition. Instead, the facility underwent a significant refit in 2018, resulting in a net capacity of 11.2 MW generated entirely from hydrothermal brine production \citep{bonafin_repowering_2019}. 

Department of Energy-funded efforts to characterize the geothermal resources in the Animas Valley, NM revealed the presence of two different thermal reservoirs: the hydrothermal resource targeted by Lightning Dock where deep geothermal fluids ascend along the Animas Valley Fault complex to $\approx$365-1000 m depth, and a secondary interval at $\approx$900-1200 m depth that requires permeability enhancement for production \citep{schochet_development_2001}. The Horquilla limestone formation defines the second reservoir, estimated to span a minimum volume of 6 cubic km based on conservative figures. By one proprietary study completed in 2001 for Ormat Technologies, a commercial geothermal company, the Horquilla has a 88\% probability of 6 MW in recoverable electricity generation potential \citep{schochet_development_2001}.

\citet{schochet_development_2001} proposed the construction of a 6 MW hybrid power plant combining hydrothermal and EGS-sourced power generation a decade before operations commenced at Lightning Dock. In their development plan, they noted several benefits of pursuing EGS in this location:
\begin{itemize}[itemsep=2pt]\label{ch4:ld_egs_support}
    \item Relatively shallow resource drives lower drilling costs
    \item EGS water requirements are attainable from paired hydrothermal operations
    \item A comprehensive initial assessment determined no significant environmental degradation is expected from geothermal operations
    \item Lightning Dock has direct access to in-place transmission lines  
    \item Opportunities exist for electricity sales to local users
    \item Purchase agreements with regional utilities are incentivized by NM legislation
\end{itemize}

As suggested by this list, the conditions at Lightning Dock offer a nearly ideal test case for an EGS proof of concept on a manageable scale. Historical land utilization in the area is primarily agricultural with few residences, so risk is low for any adverse impact on an existing population. In addition, use of a binary cycle design as proposed by \citet{schochet_development_2001} offers the potential for power production with zero GHG emissions.  

In this thesis, the \citeauthor{schochet_development_2001} concept is revisited with the existing geothermal production at Lightning Dock kept in mind; rather than building a new hybrid facility, the revised concept involves targeting the deeper reservoir as a near-hydrothermal field EGS (NF-EGS) development with a tie-back to the current Lightning Dock facility. Stepping out from the hydrothermal zone in proximity to the Animas Valley Fault complex, thermal conditions settle to a high background geothermal gradient between $\approx$ 80-120 K/km based on boreholes TG 56-14 and TG 12-7 \citep{cunniff_final_2003} -- certainly high enough to support geothermal capture. These conditions make for an interesting case study on risk mitigation options for EGS production planning.

Public records regarding power generation at Lightning Dock provide some guidance on the appropriate size for an EGS expansion. After phase 1 development, the plant produced 4 MW. An additional 6 MW was slated for phase 2, but re-powering of the plant actually added 7 MW to the capacity after several years of development stasis \citep{think_geoenergy_turboden_2020}. \citeauthor{schochet_development_2001} originally proposed a 6 MW hybrid plant for the site, but they also noted 6 MW from the Horquilla was likely understating the full reservoir potential \citeyear{schochet_development_2001}. In consideration of the step-wise trajectory of plant improvements and assessment of available thermal resource, this case study targets 5 MW as a expansion goal. 

\subsection{New Mexico Electricity Demand}

Pursuing the expansion of a power plant requires sufficient demand to ensure total revenue offsets project expenses. Fortunately, New Mexico regulations support the further development of geothermal power production in the state. Specifically, the Energy Transition Act signed in 2019 updated the New Mexico \acrlong{rps} (\acrshort{rps}) to go zero-carbon by 2050, with milestone targets along the way \citep{lillian_new_2019}. The RPS dates back to the Renewable Energy Act passed in 2004 and comes with several carve-outs, including a 30\% requirement for wind energy, 20\% for solar, and 5\% for other renewables like geothermal \citep{dsire_dsire_2021}. Public Service Company of New Mexico (PNM) is the state’s largest energy provider and services the Lordsburg area where Lightning Dock is located. PNM and Cyrq Energy currently share a 20-year \acrlong{ppa} (\acrshort{ppa}) for electricity generated at Lightning Dock. The PPA has gone through amendments over time to update both the wattage supplied to PNM and the pricing structure per MWh \citep[e.g.,][]{pnm_pnm_2014,stanfield_new_2017}. This indicates a PPA can be revisited if conditions change, which is an important aspect to consider when modeling project financials. 
In addition to the RPS requirement for a diversified renewables portfolio, coal power plants across the state face mandated shut-downs as a consequence off the Energy Transition Act. Coal currently supplies a large fraction ($\approx$ 45\%) of electric power sector consumption in New Mexico (Figure \ref{fig:nm_energy_consumption}). The supply gap introduced as coal-based production ramps down to zero could more than compensate for a 5 MW addition of no-emissions energy to the New Mexico grid.

\begin{figure}[!htp]
\centering
\includegraphics[width=\textwidth]{templates/images/Figure-EIA_NM_Energy_Consumption.pdf}
\caption[NM energy consumption]{Energy consumption by source for New Mexico. Adapted from data and graphics reported by the EIA \protect\citep{eia_new_2021}.}
\label{fig:nm_energy_consumption}
\end{figure}

\subsection{Modular Geothermal}
Limiting this expansion to a single 5 MW facility represents one design alternative, but others exist as well. One flexible option uses modular technology that recently captured the attention of high-stakes investors across the world \citep{shieber_bill_2019}. Climeon has engineered a compact binary cycle unit capable of 150 kW of generated electricity using inlet fluid temperatures rated up to 120℃ and flow rates of up to 35 kg/s \citep{climeon_climeon_2021-1}. These units can be combined into a larger deployable Power Block for 1050 kW of generated electricity \citep{winther_power_2018} (Figure \ref{fig:climeon_powerblock}). Using this technology, power plants can now be treated like multi-unit assemblages, installed all at once or over an extended period of time based on operator needs \citep{climeon_why_2018}.

\begin{figure}[!htp]
\centering
\includegraphics[width=\textwidth]{templates/images/Figure-Climeon-PowerBlock.png}
\caption[Modular power plant schematic]{Modular binary cycle power plant concept, adapted from Climeon PowerBlock schematic diagram \protect\citep{climeon_climeon_2021-1}. Each block consists of seven active units chained together for $\approx$1 MW of generating capacity.}
\label{fig:climeon_powerblock}
\end{figure}

\subsection{Flexibility with Real Options}
As discussed in Section \ref{ch2:costmod}, cost models can provide insights into the potential value gained or lost by a proposed facility before construction even begins. Well-established geothermal cost models like GETEM \citep{entingh_volume_2006} present a highly parameterized but deterministic view of cost and investment opportunity given a defined geothermal resource and development concept. Other models may apply different assumptions or mathematical treatments for various facets of the system, however they uniformly offer a single-track aspect to how the project unfolds over its lifecycle. Users can test ideas, but the solution space remains under-explored due to implicit assumptions of variable trends or static behaviors for a highly-dynamic system.  

In the cost model outlined below, the economic analysis incorporates uncertainty by replacing single value estimates with distributions for model variables. This enables the model to produce a representative range of possible outcomes when simulated many times over. In addition, the model flexibly adapts by executing real (engineering) options, where design updates triggered by changing conditions allow the system to realize upside potential or characterize the extent of downside risk. Designs need not be static, and real options can greatly increase the expected value of a project by exploring execution strategies otherwise missed by more traditional modeling approaches \citep[chap.6]{de_neufville_flexibility_2011}.

\section{Cost Model Structure}
\label{ch4:cm_structure}

Geothermal cost models typically report Levelized Cost of Electricity (LCOE) for simple comparison with other renewable energy sources. However, LCOE is standardized to represent the total lifetime costs incurred by a power plant normalized by the total lifetime power generation from start-up to plant decommissioning. LCOE is therefore not well-suited for communicating projected gains or losses under different designs or scenarios, which are the focus of this analysis. Instead, the model described here relies on \acrlong{npv} (\acrshort{npv}), a simple measure of project lifetime worth that accounts for the time value of money by applying a single interest rate, the discount rate, for both borrowing and deposits \citep[p.\ 195-215]{de_neufville_flexibility_2011}. Here, "present value" refers to a 2020 cost basis. For power generation over a 30-year lifespan -- the default for geothermal models like GETEM \citep{entingh_volume_2006} -- this basis takes the model out to 2050, a common benchmark year for future projections. 

\subsubsection{NPV Model}
\label{ch4:cm_npv}
Following the general outline for geothermal cost modeling from different sources \citep[e.g.,][]{augustine_hydrothermal_2009, beckers_introducing_2013,tester_future_2006}, this thesis considers revenue (R), operating and maintenance costs (OM), and capital expenditures (C) as the primary cash flow elements with a time dependency related to the constant discount rate (r). Capital expenses can be further decomposed into five sub-components associated with exploration, drilling \& completions, reservoir stimulation, fluid distribution, and power plant costs (Equation \ref{eq:cm_components}).

\begin{equation}
    \label{eq:cm_components}
    \begin{aligned}
    NPV &= \sum_{t=1}^{T}\frac{R_t - OM_t - C_t}{(1+r)^t}\\
    \text{where:}\\
    C_t &= \left[C_{expl} + C_{dc} + C_{stim} + C_{dist} + C_{pp}\right]_t
     \end{aligned}
\end{equation}
\\
Revenue and expenses are treated on an annual basis, meaning shorter-term fluctuations like price and production seasonality are not explicitly modeled.

\subsubsection{Exploration Costs} \label{ch4:cm_capex_expl}
Costs for exploration activities are estimated by the same method defined for the 2012 GETEM model (Equation \ref{eq:cm_cexpl}) \citep{eere_getem_2012}. 

\begin{equation}
\label{eq:cm_cexpl}
    C_{expl} = PI \cdot \left[ 1.12 \cdot (\$1\text{M} + 0.6\cdot C_{stdwell}) \right]
\end{equation}
\\
This relationship assumes slim hole (3-6" diameter) drilling for exploration at a 60\% discounted cost compared to standard-sized ($\geq$ 8.5" diameter) geothermal wells. The constant \$1M term accounts for pre-drilling costs, including field work, geophysical surveys of field structure, and interpretation of results. Technical and office support is covered by an additional 12\% applied to the estimate \citep{eere_getem_2012}. Total exploration costs are converted to a 2020 cost basis using the producer price index (PI) for electric power generation from the U.S. Bureau of Labor and Statistics \citep{us_bls_ppi_2021}.

\subsubsection{Drilling \& Completions Costs} 
\label{ch4:cm_capex_dc}

Geothermal drilling and completions costs differ from traditional oil \& gas wells due to differences in hole diameter, thermal and geochemical conditions, and the strength and abrasiveness of the target formations \citep{lowry_geovision_2017}. Here, capital expenditures for drilling rely on the cost curve described by \citet{beckers_introducing_2013}.

\begin{equation}
\label{eq:cm_cdc}
    C_{dc} = PI \cdot \left[ 1.65 \cdot 10^{-5} \cdot MD^{1.607} \right]
\end{equation}
\\
where $C_{dc}$ is measured in \$M and MD refers to well measured depth in meters. Each power plant module will require an injector-producer pair, so this represents one-half of the drilling cost per module. Drilling costs are converted to a 2020 cost basis using the industry index for electric power generation \citep{us_bls_ppi_2021}. 

Note that Equation \ref{eq:cm_cdc} was derived for well depths of 1600-9000 m. Assuming an average geothermal gradient of 100 K/km (Table \ref{tab:cm_resource_params}), the wells considered for this study could extend slightly shallower than this range, so this should be viewed as a minimum drilling \& completions estimate. The stochastic model considered later in this study includes variability in both geothermal gradient and drilling costs for a more comprehensive treatment of both variables.

\subsubsection{Simulation Costs}
\label{ch4:cm_stim}

EGS at Lightning Dock requires stimulation of the Horquilla reservoir to create fluid pathways for thermal extraction. The stimulation cost estimate used in this study comes from the recent GeoVision analysis \citep{lowry_geovision_2017}:

\begin{equation}
\label{eq:cm_capex_stim}
    C_{stim} = \$1,250,000
\end{equation}
\\
Since this represents a recent ballpark estimate, no cost basis conversion was applied in the model. In fact, the value in Equation \ref{eq:cm_stim} may be high since it includes the cost of water, which may not be a factor at Lightning Dock with the availability of hydrothermal brine from adjacent power plant operations. The model assumes stimulation is only performed for the injection well in each injector-producer pair, so this represents a \textit{per module} value.

\subsubsection{Distribution Costs}
\label{ch4:cm_capex_dist}

Fluid distribution costs include the entire surface piping system between the wells and power plant modules. This study uses the same estimate included in the GEOPHIRES model \citep{beckers_introducing_2013}.

\begin{equation}
\label{eq:cm_dist}
    C_{dist} = \$50,000 \cdot q
\end{equation}
\\
where $q$ is the thermal power of the produced geofluid in MW. Under the scenario where modular power plant units are provided by a company like Climeon, fluid distribution may be included in the installation fees. Distribution capital expenditures would therefore be subsumed by power plant costs and $C_{dist}$ would reduce to zero. However, without confirmation of the fee break-down structure from Climeon, the model described here relies on Equation \ref{eq:cm_dist}. 

\subsubsection{Power Plant Costs}
\label{ch4:cm_capex_pp}

Power plant costs for a modular installation remain a source of significant uncertainty for this cost model. The GEOPHIRES model implements a temperature-variable cost estimate first described by \citet{tester_future_2006} for a binary cycle power plant \citep{beckers_introducing_2013}. \citet{schochet_development_2001} predicted produced fluid temperatures of 280-320$^\circ$F (137-160 $^\circ$C) for the Lightning Dock EGS reservoir, which equates to \$1565-\$1694 per kWh by the GEOPHIRES estimate. Converted to a 2020 cost basis \citep{us_bls_ppi_2021}, this amounts to \$2230-\$2415 per kWh.

If power plant capacity is modularized with pre-fabricated units like the Climeon PowerBlock concept, economies of scale should reduce the cost of construction and installation. Unanswered company inquiries left this rationale unconfirmed. Nevertheless, the author chose to assume a round-number estimate accounting for a modularity discount (Equation \ref{eq:cm_pp}). This could easily be replaced by more accurate numbers when those values become available.

\begin{equation}
\label{eq:cm_pp}
    C_{pp} = \$2,000 \cdot q
\end{equation}
\\
The value in Equation \ref{eq:cm_pp} represents a 2020 estimate. Pump costs are assumed to be included in this expense.

\subsubsection{Operations and Maintenance Expenses}
\label{ch4:cm_opex}

Operations and Maintenance expenditures can similarly be subdivided into subsidiary costs, including those related to the power plant, wells, and water management \citep{beckers_introducing_2013}.

\begin{equation}
    \label{eq:cm_opex}
    OM_t = \left[C_{OM,pp} + C_{OM,well} + C_{OM,water}\right]_t
\end{equation}


\subsubsection{Model Parameters}
In order to estimate the values of these components, the following parameters were defined for a deterministic cost model. The values selected are reflective of the Animas, NM region, the Lightning Dock facility, and limits on components of the system to the best of the author's knowledge.
\\
\\
The following parameters relate to resource recovery:
\begin{table}[!htp]
\centering
\resizebox{\textwidth}{!}{
\begin{tabular}{|l|c|l|}
\hline
\multicolumn{1}{|c|}{\textbf{Parameter}} & \textbf{Value} & \multicolumn{1}{c|}{\textbf{Reference/Notes}} \\ \hline
Ambient surface temperature & 15.8 $^\circ$C & \citep{dahal_evaluation_2012} \\ \hline
Average geothermal gradient & 100 K/km & \citep{crowell_history_2014} \\ \hline
Produced fluid temperature & 120 $^\circ$C & Limit on inlet temperature \citep{climeon_climeon_2021-1} \\ \hline
Flow rate per unit & 35 kg/s & Limit on unit flow rate
\citep{climeon_climeon_2021-1} \\ \hline
Cooling in production well & 5 $^\circ$C & \citep[based on][]{beckers_introducing_2013, entingh_volume_2006} \\ \hline
Thermal drawdown rate & 0.5\% & \citep{entingh_volume_2006} \\ \hline
Water loss rate & 2\% & \citep{freeman_system_2018} \\ \hline
\end{tabular}}
\caption[Cost model parameters for resource recovery]{Parameters related to resource recovery in the cost model}
\label{tab:cm_resource_params}
\end{table}
\\
The following parameters relate to capital expenditures, spanning exploration, drilling \& completions, power plant costs, distribution, and stimulation:
\begin{table}[!htp]
\centering
\resizebox{\textwidth}{!}{
\begin{tabular}{|l|c|l|}
\hline
\multicolumn{1}{|c|}{\textbf{Parameter}} & \textbf{Value} & \multicolumn{1}{c|}{\textbf{Reference/Notes}} \\ \hline
Drilling costs & \$1,306,000 & \citep{dahal_evaluation_2012} \\ \hline
Surface plant costs & 100 K/km & \citep{crowell_history_2014} \\ \hline
Stimulation costs (S) & 120 $^\circ$C & Limit on inlet temperature \citep{climeon_climeon_2021-1} \\ \hline
Fluid distribution costs (D) & 35 kg/s & Limit on unit flow rate
\citep{climeon_climeon_2021-1} \\ \hline
Redevelopment discount factor & 5 $^\circ$C & \citep[based on][]{beckers_introducing_2013, entingh_volume_2006} \\ \hline
\end{tabular}}
\caption[Cost model parameters for power plant CAPEX]{Parameters related power plant CAPEX in the cost model.}
\label{tab:cm_capexpp_params}
\end{table}
\chapter{Analytics Application}\label{ch5:expl_applied}

\section{Data Modeling}

\subsection{Classification Algorithms}

\subsubsection{Logistic Regression}

\subsubsection{Decision Trees}\label{ch5:decision_trees}

\subsubsection{Tree Ensembles}

\subsubsection{Neural Networks}

\subsection{Uncertainty Analysis}

\subsubsection{Bootstrap Estimation}

\subsubsection{Information Entropy}

\subsubsection{Bayesian Networks}

%\chapter{Power Plant Cost Model Results}\label{ch6:cm_results}

Chapter \ref{ch4:cm_prep} outlined the cost modeling strategy for a hypothetical 5 MW expansion project of the Lightning Dock power plant in Animas Valley, NM. This chapter reviews the results of the different model approaches, explores insights gained from those models, and describes how this approach mitigates risks associated with geothermal production.

\section{Static Model}
\label{ch6:static_mod}

\subsection{Model Selection}

Section \ref{ch4:cm_rev} described the use of brine effectiveness in the cost model for determining the power output of a binary cycle plant for a given production temperature and flow rate. This formulation provides a choice of how to manage the cost model mechanics due to a trade-off between plant capacity and flow rate for a given brine effectiveness (Equation \ref{eq:cm_rev}).

In addition, installation of the Lightning Dock expansion can take place over a variety of different deployment schedules due to the modularity of the system. Rather than drill ten wells and install five binary cycle plants all at once, delaying aspects of the installation can be financially beneficial and less of an initial risk for the project.  

Figure \ref{fig:static_model_compare} shows the results for the pre-set capacity and pre-set flow rate static models for sixty (60) installation schedule permutations. The pre-set capacity model results in project losses of \$20 million or more for all tested installation options. On the other hand, the fixed flow-rate model only drops below \$0 NPV for a handful of project plans, achieving \$3.7 million NPV for the case marked with a red diamond where three (3) modules are installed up front and two (2) additional ones go live after a year of operation. Based on these results, all cost models used in this thesis apply a fixed flow rate per production well and derive electricity generation numbers based on the temperature of the produced brine.

\begin{figure}[!htp]
\centering
\includegraphics[width=\textwidth]{templates/images/Figure-Static_Model_Construction.png}
\caption[Static cost model comparison]{Static cost model comparison between pre-set aggregate capacity (5 MW target, green) and pre-set flow rate per production well (40 kg/s, purple), plotted against module installation schedule. Both models deploy five modules in all schedule permutations involving up to a five-year period. The red diamond marks the optimal model and power plant expansion plan.}
\label{fig:static_model_compare}
\end{figure}

\subsection{Construction Optimization}

After lifted the fixed-capacity requirement after reviewing results in Figure \ref{fig:static_model_compare}, the hypothetical power plant modules being modeled have an predicted output of 2.1 MW based on the resource and production parameters defined in Section \ref{ch4:cm_npv}. This reduces the required module installation count to a total of three (3) modules based on the original expansion target of $\approx$5 MW. Table \ref{tab:static_optimization} revisits the installation schedule grid search exercise to determine the optimal project plan under these circumstances. At an NPV of \$1.0 million, the best option deploys two (2) modules initially and adds an additional one (1) at the end of the first year. In order to standardize cost models for direct comparison, this installation plan is used for all cost models throughout the rest of this analysis.

\begin{table}[!htp]%{R}{0.4\linewidth}
\centering
\begin{tabular}{|c|c|c|c|}
\hline
\textbf{Year 0} & \textbf{Year 1} & \textbf{Year 2} & \textbf{NPV (\$M)} \\ \hline
3 & 0 & 0 & -\$1.1 \\ \hline
1 & 0 & 2 & -\$0.3 \\ \hline
1 & 1 & 1 & \$0.5 \\ \hline
1 & 2 & 0 & \$0.6 \\ \hline
2 & 0 & 1 & \$0.6 \\ \hline
2 & 1 & 0 & \$1.0 \\ \hline
\end{tabular}
\caption[Static model module installation schedule]{Grid search for the optimal power plant installation schedule based on the static cost model. Values are in \$M, where M is millions.}
\label{tab:static_optimization}
\end{table}

\subsection{Statistics}

\begin{table}[!htp]
\centering
\begin{tabular}{|l|c|}
\hline
\textbf{Static Model Statistics} & \textbf{\$M} \\ \hline
NPV & \$1.0 \\ \hline
\end{tabular}
\caption[Static model statistics]{Static model statistics. NPV is reported in \$M, where M is millions.}
\label{tab:static_mod_stats}
\end{table}

\section{Probabilistic Model}



Common methods for evaluating the ensemble include building an NPV histogram, constructing a cumulative distribution function (target curve), and averaging the results together for Expected Value of NPV (ENPV). Other interesting metrics for model comparison include standard deviation of NPV, extreme cases like P$_{05}$ and P$_{95}$ results, and a direct comparison to the deterministic NPV (NPV$_{det}$). These 



\section{Recap}
%\chapter{Discussion}\label{ch7:discuss}
\label{ch7:discussion}

Risk acts as a significant barrier to the adoption of geothermal energy as part of a larger energy portfolio for commercial oil \& gas companies. Here, the word \textit{risk} refers to the "potential inability to achieve overall program objectives within defined constraints" and can be characterized by a probability of occurrence and consequence of failure \citep{malone_development_2004}.  Companies considering investments in geothermal want to minimize risk exposure, so strategies to mitigate this risk will naturally act as enablers for investment in geothermal during the on-going energy transition.

Measure of the potential inability to achieve overall program objectives within
defined constraints and has two components: 1) the probability/likelihood of failing
to achieve a particular outcome, and 2) the consequence/impact of failing to achieve
that outcome

\section{Field Lifecycle}
\label{ch7:field_lifecycle}

Maturing a geothermal asset from initial concept through site decommissioning represents a complex project lifecycle spanning up to several decades in length. Figure \ref{fig:geothermal_field_lifecycle} illustrates the decomposition of a geothermal field lifecycle into a level 1 process flow that mimics that of a typical hydrocarbon field. The level 2 decomposition describes a work breakdown structure, each step with its own inherent risks. Here, the primary play risk elements for geothermal introduced in Section \ref{ch2:sysfund} have been reframed as four components: heat, permeability, fluids, and seal. Each appear in both the Exploration and Appraisal phases of the project. 

\begin{figure}
\centering
\includegraphics[width=\textwidth]{templates/images/Figure-SystemDecomposition.png}
\caption[Geothermal field lifecycle]{Proposed geothermal field lifecycle and two levels of decomposition defining major project phases and a high-level WBS. Dotted lines indicate where machine learning methods could mitigate risk in exploration and appraisal. Dashed lines depict where cost models might mitigate risk during the development and production phases.}
\label{fig:geothermal_field_lifecycle}
\end{figure}

The red dotted outline in Figure \ref{fig:geothermal_field_lifecycle} illustrates activities in the Exploration and Appraisal stages where machine learning methods described in Chapters \ref{ch3:expl_prep} and \ref{ch5:expl_ml} can reduce the overall risk profile. Geothermal exploration commonly focuses on areas where data and known resources are already present. Reviewing available data to identify feature relationships suggestive of favorable locations is an early pre-screening activity for mitigating the risk of costly exploration failures \citep{doughty_geovision_2018}. ML algorithms described in Chapter \ref{ch5:expl_ml} provide data-driven methods for uncovering these complex feature relationships and generating resource favorability maps rapidly and at low cost. Furthermore, feature importances derived from recursive feature elimination (Section \ref{ch5:logreg_rfe}), impurity measures like Gini index or entropy (Section \ref{ch5:impurity}), or Shapley analysis (Section \ref{ch5:xgb_shapley}) directly rank different data sources by their value for predictive modeling. These measures could also be used to guide exploration and appraisal spending on additional data purchases or acquisition efforts. For example, recognizing that silica geothermometer temperatures, heat flow measurements, crustal thickness, and density of volcanic dikes and springs all highly influence the geothermal gradient classification model (see Section \ref{ch5:xgb_shapley}), an exploration team could focus time and budget on 1) field surveys for silica concentration sampling, 2) field or remote-sensing mapping of springs and dikes, and 3) seismic acquisition for improved crustal thickness estimates where those estimates are least well-constrained. As suggested by the black dotted line, the learnings from assessing geothermal heat content with machine learning techniques in Chapter \ref{ch5:expl_ml} are easily transferable to assessments for the other risk elements. 

Cost modeling similarly offers benefits for risk mitigation in the geothermal project lifecycle, as illustrated with the dashed lines in Figure \ref{fig:geothermal_field_lifecycle}. Surface plant construction and drilling activities take place during the development phase and continue into production as thermal decline or market forces trigger field management responses. Rather than treat the extent of these activities as known unknowns, characterizing and including them in flexible economic models offers the opportunity to assess their impact and scenario-test for the optimal field strategy. As the analysis in Chapter \ref{ch6:cm_results} showed, models can include local uncertainties, e.g., geothermal gradient or decline rate, as well as broader risks like a carbon tax or national electrification. The red long-dashed line in Figure \ref{fig:geothermal_field_lifecycle} surrounds factors considered by the cost model in Chapters \ref{ch4:cm_prep} and \ref{ch6:cm_results}. The gray dashed line depicts additional aspects of the development and production phases that could also be characterized with distributions and decision rules in determining project viability or refining field strategy.

\section{Role of Uncertainty}
\label{ch7:uncertainty_role}

Uncertainty exists, as does the opportunity to include it in a larger decision-making process for geothermal adoption. In the exploration phase, feature standard errors and maps of entropy --- or another measure of collective uncertainty --- can influence project choices. Observing pervasively high standard errors for a data layer (see Section \ref{ch5:measure_uncertainty}) raises the question of whether that data should be re-acquired using different tools or survey methods, or if better quality data might already be available for use or purchase. And zones of high entropy in measurement uncertainty highlight areas that need additional attention. Is the entropy a result of insufficient data to train machine learning models, leading to poor discrimination ability for the dependent variable classes? Are the data in areas of high entropy inconclusive or poorly conditioned? Or should more data be acquired using part of the exploration and appraisal budget? Pursuing these questions helps frame a refined project plan for the early phases of the field lifecycle. Data-driven choices can be made for prioritizing time and resource allocations to data science and data engineering (using existing data), field studies and data acquisition (supplementing existing data), or pivoting to alternative areas with greater certainty for development efforts.

If an ensemble of models is considered, structural or model uncertainty (see Section \ref{ch5:model_uncertainty}) can direct efforts around how to approach machine learning prediction. For example, when entropy appears high throughout the area of interest (AOI), and most of the models show reasonable agreement except for one or two, this could justify down-selecting those models and re-evaluating from the reduced ensemble. But if high variance is observed across many models, this might indicate the input data insufficiently describes the system. Likewise, predictions from probabilistic models that show high parameter uncertainty in the AOI (Section \ref{ch5:param_uncertainty}) should be treated as suspect and the model refactored or retrained on a larger data set. Insights like these mitigate the risk of over-confidence in under-performing models, and hence avoiding a poor exploration well placement based on those models further down the line.

For cost models, directly incorporating uncertainty serves to counter the classic misconception that using average values for a complex system will lead to an average system result \citep[Flaw of Averages,][p.\ 17-19]{de_neufville_flexibility_2011}. Applying variable distributions and generating expected values from multiple model realizations provides accurate median estimates and describes the spread of potential results. As discussed in Section \ref{ch6:cost_model_metrics}, target curves and percentiles defining Value at Risk and Value at Gain offer a richer set of metrics for model comparison. Using these metrics can reveal the combination of strategic choices for a geothermal project timeline and execution that mitigate the risk of project losses and captures to most potential gain.

\section{Risk Mitigation Analysis}
\label{ch7:risk_mitigation}

NASA approaches project risk mitigation by assigning likelihood (Table \ref{tab:likelihood_table}) and consequence (Table \ref{tab:consequence_table}) values to known risks \citep{malone_development_2004}. This begins with constructing a risk log as shown in Table \ref{tab:risk_log}. Even the act of constructing this table adds value to a project team by building alignment on group judgment and assumptions regarding risk relevance and potential impact.

\begin{table}[!htp]
\resizebox{\textwidth}{!}{
    \begin{tabular}{|l|l|c|c|c|}
    \hline
    \multicolumn{1}{|c|}{\textbf{ID}} & \multicolumn{1}{c|}{\textbf{Description}} & \textbf{Likelihood} & \textbf{Consequence} & \textbf{Risk} \\ \hline
    EXP1 & Insufficient exploration budget & 4 & 3 & 12 \\ \hline
    EXP2 & Poor subsurface characterization & 4 & 4 & 16 \\ \hline
    EXP3 & Permitting delays & 3 & 3 & 9 \\ \hline
    DRL1 & Rig unavailability & 4 & 3 & 12 \\ \hline
    DRL2 & Drilling cost overruns & 3 & 3 & 9 \\ \hline
    DRL4 & Downhole equipment failure & 3 & 2 & 6 \\ \hline
    PRD1 & Insufficient production budget & 3 & 4 & 12 \\ \hline
    PRD4 & Rapid thermal decline & 3 & 4 & 12 \\ \hline
    PRD5 & Demand variability & 2 & 4 & 8 \\ \hline
    PRD6 & Wrong-sized infrastructure & 3 & 4 & 12 \\ \hline
    \end{tabular}
}
\caption[Geothermal risk log]{Subset of geothermal project risks, each assigned a likelihood of occurrence (see Table \ref{tab:likelihood_table}) and consequence (see Table \ref{tab:consequence_table}). Risk is likelihood $\times$ consequence.}
\label{tab:risk_log}
\end{table}

\begin{table}[!htp]
\centering
\begin{tabular}{|c|l|c|}
\hline
\multicolumn{3}{|c|}{\textbf{Likelihood}} \\ \hline
\textbf{Score} & \multicolumn{2}{c|}{\textbf{Likelihood of Occurrence (p)}} \\ \hline
5 & Near certainty & (\ 0.8,1.0\ {]} \\ \hline
4 & Highly likely & (\ 0.6,0.8\ {]} \\ \hline
3 & Likely & (\ 0.4,0.6\ {]} \\ \hline
2 & Low likelihood & (\ 0.2,0.4\ {]} \\ \hline
1 & Not likely & {[}\ 0.0,0.2\ {]} \\ \hline
\end{tabular}
\caption[Risk likelihood]{Likelihood score based on probability of occurrence (p). Adapted from NASA S3001 Guidelines for Risk Management, v.G \protect\citep{malone_development_2004}.}
\label{tab:likelihood_table}
\end{table}

\begin{table}[!htp]
\resizebox{\textwidth}{!}{
    \begin{tabular}{|l|P{0.17\linewidth}|P{0.17\linewidth}|P{0.17\linewidth}|P{0.17\linewidth}|P{0.17\linewidth}|}
    \hline
     & \multicolumn{5}{c|}{\textbf{Consequence}} \\ \hline
     & 1 & 2 & 3 & 4 & 5 \\ \hline
    \textbf{Performance} & Minimal to no impact on meeting project goals. & Minor impact on meeting project goals. & Unable to meet a specific project goal, but remaining goals can be achieved. & Unable to meet multiple project goals, but minimum project success still achievable. & Unable to meet multiple project goals, minimum project success not achievable. \\ \hline
    \textbf{Schedule} & Minimal to no impact on project schedule. & No change to critical milestones; at least 10-day buffer between any delay and impact on milestone timing. & No change to critical milestones; less than 10-day buffer between any delay and impact on milestone timing. & One or more critical milestones slip, impacting overall project schedule. & One or more critical milestones slip; one or more milestones cannot be achieved. \\ \hline
    \textbf{Cost} & Minimal to no impact on cost. & Minor impact on cost. Variance $< 5\%$ of total approved budget. & Impact on cost. Variance $> 5\%$ but $\leq 10\%$ of total approved budget. & Impact on cost. Variance $> 10\%$ but $\leq 15\%$ of total approved budget. & Major impact on cost. Variance $> 15\%$ of total approved budget. \\ \hline
    \end{tabular}
}
\caption[Risk consequence]{Consequence score defining the scale of impact a risk might have if it becomes a reality. Adapted from NASA S3001 Guidelines for Risk Management, v.G \protect\citep{malone_development_2004}.}
\label{tab:consequence_table}
\end{table}

Having captured the key risks for a geothermal project, appropriate mitigation methods addressing those risks are also defined (Table \ref{tab:mitigation_log}). The project team can then evaluate and assign scores for likelihood and consequence to the mitigated risks, calculate the remaining risk value, and show the risk reduction associated with the mitigation.

The NASA methodology plots the risks before and after mitigation on a $5\times5$ matrix, where high-likelihood, high-consequence risks fall in the upper right and the lower left represents the ideal low-likelihood, minimal consequence region. Figure \ref{fig:risk_matrix} illustrates this method by plotting risks from Tables \ref{tab:risk_log} (gray) and \ref{tab:mitigation_log} (white), mapping between the two with arrows. The risk matrix serves two main functions. First, it quickly differentiates between risks by placing the most-probable and problematic ones closer to top right, visually depicting a risk prioritization. Secondly, different mitigation strategies for the same risk can be evaluated and plotted in the same $5\times5$, and the optimal choice will be the one landing closest to the lower left.

\begin{table}[!htp]
\renewcommand{\arraystretch}{2.5}
\resizebox{\textwidth}{!}{
    \begin{tabular}{|l|M{0.12\linewidth}|M{0.12\linewidth}|M{0.12\linewidth}|M{0.12\linewidth}|P{0.42\linewidth}|}
    \hline
    \multicolumn{1}{|c|}{ID} & Mitigated Likelihood & Mitigated Consequence & Mitigated Risk & Risk Reduction & Mitigation Action \\ \hline
    EXP1 & 1 & 2 & 2 & 10 & \cellcolor[HTML]{E7E6E6}Use ML model predictions to reduce costs in exploration \\ \hline
    EXP2 & 1 & 3 & 3 & 13 & \cellcolor[HTML]{E7E6E6}Use ML for baseline model, acquire data based on importances \\ \hline
    EXP3 & 2 & 3 & 6 & 3 & Engage with regulators early to fast-track permitting \\ \hline
    DRL1 & 3 & 3 & 9 & 3 & Secure rig early and pre-plan for future drilling needs \\ \hline
    DRL2 & 1 & 3 & 3 & 6 & \cellcolor[HTML]{E7E6E6}Use ML to identify high gradient areas faster, shallower drilling \\ \hline
    DRL4 & 2 & 2 & 4 & 2 & Use service companies with high-Temp equipment track record \\ \hline
    PRD1 & 1 & 3 & 3 & 9 & \cellcolor[HTML]{E7E6E6}Cost modeling for optimized spending and return \\ \hline
    PRD4 & 3 & 2 & 6 & 6 & Regularly re-drill wells or re-stimulate reservoir \\ \hline
    PRD5 & 2 & 2 & 4 & 4 & \cellcolor[HTML]{E7E6E6}Flexibility in power generation based on market \\ \hline
    PRD6 & 1 & 2 & 2 & 10 & \cellcolor[HTML]{E7E6E6}Flexible design for demand-triggered capacity changes \\ \hline
    \end{tabular}}
\caption[Geothermal mitigation log]{Proposed mitigations for risks listed in Table \ref{tab:risk_log} and the change in risk associated with those mitigations. Risk values are likelihood x consequence. Shaded mitigations include one of the options described in this thesis and appear in Figure \ref{fig:risk_matrix}. Adapted from NASA S3001 Guidelines for Risk Management, v.G, which also includes scoring break-downs for human and asset safety (Malone \& Moses, 2004). }
\label{tab:mitigation_log}
\end{table}


\begin{figure}[htp]
\centering
\includegraphics[width=\linewidth]{templates/images/Figure-Risk_Matrix.png}
\caption[Risk matrix]{Risk matrix for categorizing and prioritizing project risks (dark gray) and charting risk mitigation strategies (white). Marker size corresponds with risk value (likelihood $\times$ consequence). Risk labels match those in Table \ref{tab:full_risk_log}.}
\label{fig:risk_matrix}
\end{figure}




\section{Proposed Strategies}
%\chapter{Conclusions}\label{ch8:conclusions}

%\chapter{Future Work}\label{ch9:future_work}

%\appendix
%\chapter{Appendix A: Exploration}
\label{app:a}

\section{Script 1: Data Prep}

\section{Script 2: }

\section{Script 3: }

\section{Script 4: }

\section{Script 5: }

\section{Script 6: }

\section{Script 7: }
%\chapter{Cost Model Spreadsheets}\label{app:B_cost_model_spreadsheets}

\section{Static NPV Model}\label{app:B_static_model}
The static model described in Section \ref{ch4:cm_concept} was implemented in Microsoft Excel as a single worksheet for cost analysis. Figures \ref{fig:static_model_sheet1} and \ref{fig:static_model_sheet2} show the model when flow rate is pre-defined and capacity depends on the input temperature of the produced brine. Not shown is the supporting look-up table for the EIA STEO-based electricity price forecast (Figure \ref{fig:electricity_pricing}).
%\vfill
%\pagebreak

\begin{figure}[H]
\centering
\includegraphics[width=\textwidth]{templates/images/Figure-Static_Model_SheetA.pdf}
\caption[Static cost model spreadsheet (part 1)]{First part of static NPV cost model spreadsheet for the geothermal expansion project. Values in gray cells with orange font are calculated using inputs from the rest of the sheet.}
\label{fig:static_model_sheet1}
\end{figure}

\begin{figure}[H]
\centering
\includegraphics[width=\textwidth]{templates/images/Figure-Static_Model_SheetB.pdf}
\caption[Static cost model spreadsheet (part 2)]{Second part of static NPV cost model spreadsheet for the geothermal expansion project. Parameters in gray cells with orange font are calculated using inputs from the rest of the sheet. The orange cells in the annual cash flow analysis are manual entry fields for constructing power-plant modules. The analysis only extends out to year 2 for visualization purposes, but continues to year 30 in the complete spreadsheet.}
\label{fig:static_model_sheet2}
\end{figure}

%\pagebreak
\section{Probabilistic NPV Models}\label{app:B_flex_models}
The probabilistic NPV model was implemented as an extension of the static NPV model in Excel, with variable look-ups using the PDFs described in Section \ref{ch4:pdfs}. Flexible design options described in Section \ref{ch4:flex_design_options} were implemented as decision rules in the cash flow analysis. Figures \ref{fig:probabilistic_model_sheet1} and \ref{fig:probabilistic_model_sheet2} illustrate the spreadsheet for the Full Flexibility case (see Section \ref{ch4:flex_reduce_case}). The results histogram, target curve, and summary statistics were generated using a 2000-row data table tied to the NPV calculation cell (not shown).
%\vfill
%\pagebreak

\begin{figure}[H]
\centering
\includegraphics[width=\textwidth]{templates/images/Figure-Flexible_Model_SheetA.pdf}
\caption[Probabilistic cost model spreadsheet (part 1)]{First part of probabilistic NPV cost model spreadsheet for the geothermal expansion project. Values in gray cells with orange font are calculated using inputs from the sheet or distributions in other worksheets. PDF look-ups are implemented for Average Geothermal Gradient, Initial Average Reservoir Temperature, Drilling \& Completions Costs, Thermal Drawdown Rate, and Price Forecast.}
\label{fig:probabilistic_model_sheet1}
\end{figure}

\begin{figure}[H]
\centering
\includegraphics[width=\textwidth]{templates/images/Figure-Flexible_Model_SheetB.pdf}
\caption[Probabilistic cost model spreadsheet (part 2)]{Second part of probabilistic NPV cost model spreadsheet for the geothermal expansion project. Parameters in gray cells with orange font are calculated using inputs from the rest of the sheet. PDF look-ups are implemented for Average Geothermal Gradient, Initial Average Reservoir Temperature, Drilling \& Completions Costs, Thermal Drawdown Rate, and Price Forecast. The orange cells in the annual cash flow analysis are manual entry fields for constructing power-plant modules. Decision rules are implemented in the annual cash flow section. The yearly breakdown of cost and revenue only extends out to year 2 for visualization purposes, but continues to year 30 in the full spreadsheet.}
\label{fig:probabilistic_model_sheet2}
\end{figure}
\include{biblio}
\end{document}
